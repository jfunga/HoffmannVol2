% !TEX root = ../../main.tex
% =============================
% Intro del capítulo
% =============================

% --- Subtítulo ---
\subtitulocapitulo{Cada exponente, un salto hacia lo infinito}

\vspace{2em} %

% --- Reseñas históricas ---
\section*{Reseña historica}
\begin{reseñaplana}
El célebre problema del ajedrez y el trigo:
Se cuenta que un sabio pidió como pago un grano de trigo en la primera casilla de un tablero de ajedrez, 
luego el doble en la segunda, y así sucesivamente hasta la casilla 64. El rey aceptó pensando que era poco, 
pero la cantidad total resultó $2^{64}$ - 1 granos, una cifra astronómica imposible de pagar.
\textit{Para dimensionarlo:} Serían unos 18 millones de billones de granos de trigo, eso es mucho más trigo 
del que podría producir la Tierra en varios siglos.
Esta historia ilustra el crecimiento exponencial de la potenciación.
\end{reseñaplana}

% --- Importancia actual ---
\section*{Importancia actual}
\begin{reseñaplana}
La potenciación es la clave para entender el crecimiento acelerado y 
fenómenos que se multiplican rápidamente. Gracias a ella podemos predecir, controlar y 
optimizar procesos en ciencia, salud, economía y tecnología, permitiendo avances en vacunas, 
desarrollo de IA y proyecciones económicas globales.
\end{reseñaplana}

% --- Qué vas a aprender ---
\section*{¿Qué vas a aprender?}
\begin{reseñaitem}
  \item Producto de potencias de igual base
  \item Potencia de potencia
  \item Potencia de un producto
  \item Cociente de potencias de igual base
  \item Exponentes negativos
  \item Potencia de una fracción
\end{reseñaitem}

% --- Cita célebre ---
\vspace{1.5cm} % espacio arriba
\begin{flushright}
  {\fontsize{12}{14}\selectfont\itshape
  ``La potencia de un número es el eco de su multiplicación repetida hasta el infinito.''\\[6pt]
  — Hermann Hankel}%
\end{flushright}




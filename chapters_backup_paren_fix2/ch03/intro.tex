% !TEX root = ../../main.tex
% =============================
% Intro del capítulo
% =============================

% --- Subtítulo ---
\subtitulocapitulo{El puente entre enteros y decimales}

\vspace{2em} %
% --- Reseñas históricas ---
\section*{Reseña historica}

\begin{reseñaplana}
La tradición cuenta que el pitagórico Hipaso de Metaponto (s. V a.C.) fue castigado e incluso arrojado al 
mar por revelar la existencia de los números irracionales (la raíz de 2). Para los pitagóricos, que basaban su filosofía 
en la armonía de los números enteros y racionales, aceptar irracionales era casi una “herejía matemática”.
\end{reseñaplana}

% --- Importancia actual ---
\section*{Importancia actual}

\begin{reseñaplana}
Los números irracionales amplían el universo numérico y permiten modelar fenómenos naturales, 
físicos y económicos con precisión.  Sin ellos, no tendríamos conceptos tan vitales como el círculo perfecto, 
la probabilidad continua, ni ecuaciones diferenciales que describen la expansión del universo o el crecimiento de una inversión bancaria.
\end{reseñaplana}

% --- Qué vas a aprender ---
\section*{¿Qué vas a aprender?}
\begin{reseñaitem}
  \item ¿Qué es una razón o relación?
  \item Números racionales e irracionales
  \item Conoce a $\pi$
  \item Ver irracionales en raíces
  \item Otros irracionales famosos
  \item Esquema numérico
\end{reseñaitem}

% --- Cita célebre ---
\vspace{1.5cm} % más aire arriba
\begin{flushright}
  {\fontsize{12}{14}\selectfont\itshape
  ``Los números irracionales son la prueba de que la naturaleza no siempre cabe en la medida humana.''\\[6pt]
  — David Hilbert}%
\end{flushright}



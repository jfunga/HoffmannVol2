% !TEX root = ../../main.tex
% =============================
% Intro del capítulo
% =============================

% --- Subtítulo ---
\vspace{0.5cm}
\subtitulocapitulo{Descifrando el poder oculto de las raíces.}

\vspace{1em}
% --- Reseñas históricas ---
\section*{La tablilla que calculaba.}

\begin{reseñaplana}
Hace unos 3{,}800 años, en algún lugar de la antigua Mesopotamia, un aprendiz babilonio —posiblemente 
un joven no mucho mayor que tú— estaba sentado frente a una pequeña tablilla de arcilla húmeda. 
Tenía un estilete en la mano (una especie de lápiz hecho de caña) y estaba aprendiendo matemáticas.
En vez de escribir con números como los nuestros, él usaba una base diferente: el \textbf{sistema sexagesimal}, 
con base 60. Es decir, contaban como nosotros medimos el tiempo: 60 segundos, 60 minutos…
Ese día, el joven trazó un cuadrado con sus diagonales y escribió junto a él algo que parecía un número extraño:
\[
1;24,51,10
\]
A simple vista no parece gran cosa, pero si hoy convertimos ese número a nuestro sistema decimal, obtenemos:
\[
1.414213\dots
\]
¡una aproximación increíblemente precisa de la raíz cuadrada de 2!
Solo seis milésimas de diferencia con el valor real que conocemos hoy con calculadoras.
Esa tablilla, hecha de arcilla cocida, sobrevivió miles de años enterrada bajo la arena. Fue encontrada siglos
después y hoy se conserva en el \textit{Museo de Arte de la Universidad de Yale} (Estados Unidos), con el 
nombre \textbf{YBC 7289}.
Está guardada en una vitrina, pequeña y color marrón claro, con inscripciones cuneiformes 
—esas marcas en forma de cuña que usaban los babilonios para escribir.
Cuando la miras de cerca, no ves solo una piedra antigua: ves la huella de un estudiante que hace casi cuatro 
milenios ya entendía lo que es el número irracional más famoso de todos.
Y si lo piensas, hay algo mágico en eso:
Un chico del otro lado del mundo, hace miles de años, resolvió el mismo tipo de problema que tú podrías 
resolver hoy en clase, sin calculadora, sin papel moderno, solo con su mente y una tablilla de barro.
Así, la tablilla \textbf{YBC 7289} no es solo una pieza de museo; es una puerta en el tiempo que nos recuerda 
que la curiosidad y el ingenio matemático han estado dentro del ser humano desde siempre.
\end{reseñaplana}

\section*{La raíz cuadrada que cambió los videojuegos}

\begin{reseñaplana}
En 1999, el videojuego \textit{Quake III Arena} cambió la historia de la programación con un truco casi mágico. 
Para que sus escenas 3D se vieran realistas, el juego necesitaba calcular millones de veces por segundo el valor 
de $ \frac{1}{\sqrt{x}} $ —una operación esencial para la luz, las sombras y el movimiento. Los programadores 
inventaron un método tan ingenioso como misterioso: el \textit{fast inverse square root}, un algoritmo que usaba 
un ``número mágico'' llamado \texttt{0x5f3759df} para obtener una estimación increíblemente rápida, corrigiéndola 
luego con una fórmula de Newton.  
Gracias a ese truco, \textit{Quake III} se volvió uno de los juegos más fluidos y avanzados de su tiempo. 
Hoy los procesadores modernos ya realizan ese cálculo de forma automática, pero el algoritmo quedó en la historia 
como una leyenda de la era dorada de la programación, una muestra de cómo la creatividad humana puede encontrar belleza
y eficiencia incluso en el lenguaje invisible de las matemáticas.
\end{reseñaplana}

\section*{¿Qué vas a aprender?}
\begin{aprende}
  \item Comprender qué significa obtener la raíz de un número y cómo se relaciona con las potencias.  
  \item Reconocer el signo correcto de una raíz y el papel de los exponentes fraccionarios.  
  \item Calcular raíces de productos, cocientes y radicales encadenados con precisión.  
  \item Simplificar radicales reduciendo índices, extrayendo o introduciendo factores según corresponda.  
  \item Operar con radicales semejantes y de distinto índice en sumas, restas y multiplicaciones.  
  \item Aplicar las propiedades de los radicales para resolver expresiones y problemas algebraicos con mayor facilidad.  
\end{aprende}

\vspace{1.5cm} % más aire arriba
\begin{flushright}
  {\fontsize{12}{14}\selectfont\itshape
  ``La raíz cuadrada fue el camino por el cual el hombre descubrió\\ 
  que no todo puede expresarse como fracción.''\\[6pt]
  — Morris Kline}%
\end{flushright}

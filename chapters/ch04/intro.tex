% !TEX root = ../../main.tex
% =============================
% Intro del capítulo
% =============================

% --- Subtítulo ---
\vspace{0.5cm}
\subtitulocapitulo{Del abismo de lo irracional al puente de la razón.}

\vspace{1em} %
% --- Reseñas históricas ---
\section*{Historia del Truco del Conjugado}

\begin{reseñaplana}
El truco del conjugado —multiplicar una expresión como $\tfrac{1}{\sqrt{2}-1}$ por su “reflejo” $(\sqrt{2}+1)$ 
para eliminar la raíz— nació junto con el propio descubrimiento de los números irracionales en la antigua Grecia.

Cuando los discípulos de Pitágoras descubrieron que la diagonal de un cuadrado de lado $1$ medía $\sqrt{2}$, 
comprendieron horrorizados que ese número no podía escribirse como fracción. Los pitagóricos adoraban el orden y 
la proporción, así que aquello fue casi una crisis filosófica: el caos había entrado en el mundo de los números.

En los siglos siguientes, los matemáticos griegos, especialmente Euclides (siglo III a.C.), desarrollaron métodos 
geométricos para tratar con magnitudes irracionales. Aunque no usaban símbolos como nosotros, ya manipulaban longitudes 
y áreas equivalentes a multiplicar expresiones tipo $(\sqrt{a}+b)(\sqrt{a}-b)=a-b^{2}$. Ahí está, en esencia, la primera 
aparición del truco del conjugado: una forma de “eliminar” la raíz mediante el producto de dos términos opuestos.

\textbf{Del álgebra árabe al álgebra moderna}

Siglos más tarde, en el mundo islámico medieval, matemáticos como Al-Khwarizmi (siglo IX) y Omar Khayyam (siglo XI) 
tradujeron ese razonamiento geométrico en operaciones algebraicas. Ya no hablaban solo de segmentos o áreas, sino de 
números con raíces. Para ellos, racionalizar era limpiar las expresiones, una idea de belleza matemática heredada de los griegos.

El término “racionalizar” y su uso sistemático surgieron en el Renacimiento europeo (siglo XVI), cuando el álgebra 
simbólica empezó a escribirse como hoy. Autores como François Viète (Francia) y René Descartes (en su obra \textit{La Géométrie}, 1637) 
consolidaron el truco del conjugado como una herramienta estándar.
\end{reseñaplana}

% --- Importancia actual ---
\section*{¿Por qué sigue siendo importante?}

\begin{reseñaplana}
Desde entonces, multiplicar por el conjugado se convirtió en una operación universal del álgebra, enseñada 
en todas las escuelas del mundo. Es una muestra perfecta de la belleza matemática clásica: transformar lo confuso 
en algo ordenado, usando solo simetría y lógica.

Así que, aunque hoy lo usamos casi de memoria, el truco del conjugado nació de la lucha milenaria entre lo 
racional y lo irracional, una idea que viene directamente de los pitagóricos y que sobrevivió hasta el álgebra moderna.

La racionalización es la técnica que transforma lo irracional en algo manejable, y aunque hoy existen calculadoras 
y software, sigue siendo clave para el pensamiento matemático estructurado.
\end{reseñaplana}

% --- Qué vas a aprender ---
\section*{¿Qué vas a aprender?}
\begin{aprende}
  \item Comprender qué significa racionalizar y por qué se eliminan raíces del denominador.  
  \item Racionalizar fracciones cuando el denominador tiene una raíz cuadrada.  
  \item Aplicar el proceso cuando el radical del denominador tiene un índice mayor que 2.  
  \item Usar el truco del conjugado cuando el denominador contiene una suma o resta con raíces.  
  \item Resolver casos especiales que requieren pasos adicionales o combinaciones de métodos.  
  \item Transformar expresiones complejas en sumas de radicales más simples.  
  \item Resolver ecuaciones que contienen radicales o coeficientes irracionales.  
\end{aprende}

% --- Cita célebre ---
\vspace{1.5cm} % más aire arriba
\begin{flushright}
  {\fontsize{12}{14}\selectfont\itshape
  ``Quitar la raíz del denominador es ordenar el caos en el corazón de la fracción.''\\[6pt]
  — Jorge Gid Hoffmann}%
\end{flushright}
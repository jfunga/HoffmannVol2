% !TEX root = ../../main.tex
% =============================
% Intro del capítulo
% =============================

% --- Subtítulo ---
\vspace{0.5cm}
\subtitulocapitulo{Cuando el número se convierte en coordenada y el pensamiento en dirección.}

\vspace{1em}
% --- Reseñas históricas ---
\section*{Los orígenes antiguos de los sistemas de ecuaciones.}

\begin{reseñaplana}
Mucho antes de que el álgebra moderna existiera, los antiguos matemáticos ya resolvían problemas 
que hoy reconocemos como \textbf{sistemas de ecuaciones}.  
En la antigua \textbf{China}, alrededor del siglo II a.C., se escribió una obra extraordinaria 
llamada \textit{Los Nueve Capítulos sobre el Arte Matemático} (\textit{Jiuzhang Suanshu}). En su 
capítulo octavo, titulado \textit{Fangcheng}, se presentaban métodos para resolver varios problemas 
prácticos —como repartos de granos, cálculos de áreas o intercambios comerciales— usando procedimientos 
equivalentes a los sistemas lineales que conocemos hoy.

Los matemáticos chinos organizaban los números en una tabla de bambú o sobre tableros de cálculo, representando 
cada ecuación como una fila de coeficientes. Luego aplicaban un proceso de eliminación sucesiva para simplificar 
las columnas hasta obtener las incógnitas. Era, sin saberlo, un método similar al que siglos más tarde se 
conocería como \textit{eliminación de Gauss}.  

Siglos después, el matemático \textbf{Zhu Shijie} (siglo XIII) llevó esta idea a otro nivel en su 
obra \textit{El Espejo de Jade de las Cuatro Incógnitas}, donde resolvía problemas con hasta cuatro variables. Su trabajo 
mostraba una comprensión sorprendente del álgebra simbólica, siglos antes de que esta se formalizara en Europa.

Así nació el espíritu de los sistemas de ecuaciones: \textit{una herramienta creada para coordinar múltiples verdades 
y hallar equilibrio en los números}. De los tableros de bambú a los cuadernos modernos, el objetivo sigue siendo el 
mismo: encontrar el punto donde todo encaja.
\end{reseñaplana}

\section*{Sistemas que iluminan el mundo moderno}

\begin{reseñaplana}
En el siglo XXI, los sistemas de ecuaciones siguen siendo protagonistas silenciosos del progreso tecnológico.  
Cada red eléctrica, cada antena de comunicación y cada simulador digital depende de ellos para funcionar correctamente.

Un ejemplo fascinante es el método \textbf{HELM} (\textit{Holomorphic Embedding Load-flow Method}), una técnica 
moderna para resolver los complejos sistemas no lineales que gobiernan los \textit{flujos de energía eléctrica} en las redes del mundo.  
Cada nodo de una red eléctrica puede describirse mediante ecuaciones que relacionan voltajes, corrientes y potencias.  
Resolverlas permite saber si la red está en equilibrio o al borde de un colapso.

El método HELM, desarrollado a inicios del siglo XXI, aplica principios del \textit{análisis complejo} para garantizar 
soluciones precisas, evitando los errores de los métodos iterativos tradicionales. Gracias a este tipo de avances, 
los ingenieros pueden simular con exactitud el comportamiento de sistemas eléctricos gigantescos —desde los que 
alimentan una ciudad hasta los que mantienen estable una estación espacial.

Así, lo que comenzó como una idea escrita en tablillas de bambú hoy mantiene encendida la luz del mundo.  
Los sistemas de ecuaciones no son solo un tema de clase: son la \textbf{lengua secreta del equilibrio}, el puente 
entre las civilizaciones antiguas y la tecnología moderna.
\end{reseñaplana}

\section*{¿Qué vas a aprender?}
\begin{aprende}
  \item Entender qué es un sistema de ecuaciones, cómo se representa y qué tipos de soluciones puede tener.  
  \item Aplicar los métodos de sustitución, igualación y reducción para resolver sistemas de dos ecuaciones con dos incógnitas.  
  \item Analizar sistemas que presentan incógnitas en el denominador y aprender a transformarlos para resolverlos correctamente.  
  \item Extender los métodos de resolución a sistemas con tres o más incógnitas y reconocer los casos con múltiples soluciones.  
  \item Resolver problemas reales que pueden expresarse mediante sistemas de ecuaciones, desde situaciones cotidianas hasta 
  aplicaciones científicas.  
  \item Utilizar los sistemas en contextos avanzados, como el balanceo de ecuaciones químicas o la resolución de ecuaciones 
  de segundo grado combinadas con lineales.  
  \item Explorar sistemas de grado superior o con ecuaciones irracionales, comprendiendo sus particularidades y métodos de solución.  
\end{aprende}

\vspace{1cm}
\begin{flushright}
  {\oneptup\itshape ``Resolver varias incógnitas a la vez es entender que el mundo es interdependiente.''}\\
  {\oneptup — Los Nueve Capítulos sobre el Arte Matemático (China, s. II a.C.)}
\end{flushright}


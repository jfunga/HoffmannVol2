% !TEX root = ../../main.tex% !TEX root = ../../main.tex
% =============================
% Intro del capítulo
% =============================

% --- Subtítulo ---
\vspace{0.5cm}
\subtitulocapitulo{La armonía del universo escrita en líneas y ángulos.}

\begin{center}
\fontsize{12pt}{14.4pt}\selectfont\itshape
La Geometría es probablemente el tema más antiguo y a la vez más vigente de todas las matemáticas, 
y su historia está llena de belleza, descubrimientos y significado.
\end{center}

\vspace{1em}

\section*{Cuando medir significaba sobrevivir.}

\begin{reseñaplana}
La palabra \textbf{geometría} viene del griego “geo” (tierra) y “metría” (medida): literalmente, “medir la tierra”.  
Sus orígenes se remontan al antiguo \textbf{Egipto}, donde los escribas medían los terrenos que el Nilo 
inundaba cada año para volver a repartirlos.  
Trazar líneas rectas, calcular áreas y reconstruir límites era una cuestión práctica… pero también el 
nacimiento del pensamiento geométrico.  

En la antigua \textbf{Babilonia}, los astrónomos y constructores ya conocían triángulos y proporciones; 
sabían que el cuadrado de la diagonal de un rectángulo era igual a la suma de los cuadrados de sus lados, 
mucho antes de que Pitágoras lo hiciera famoso.  
Así nació la geometría: como una mezcla de necesidad, observación y asombro ante las formas del mundo.  
\end{reseñaplana}

\section*{De la tierra al pensamiento puro.}

\begin{reseñaplana}
En el siglo VI a.C., el filósofo y matemático griego \textbf{Pitágoras} fundó una escuela que veía 
en los números la esencia del universo.  
Para ellos, los triángulos, cuadrados y círculos no eran solo figuras: eran armonías visibles.  
Más tarde, \textbf{Euclides}, en Alejandría, escribió los famosos \textit{Elementos}, una obra monumental 
que organizó toda la geometría conocida en definiciones, axiomas y teoremas.  

Durante más de dos mil años, el libro de Euclides fue el manual por excelencia del razonamiento lógico.  
De sus páginas nacen teoremas que todavía enseñamos hoy: el de Pitágoras, las proporciones de Tales, 
las propiedades del círculo, las rectas paralelas...  
La geometría se convirtió en el lenguaje de la lógica visual, el entrenamiento de la mente matemática.  
\end{reseñaplana}

\section*{La geometría que da forma al mundo.}

\begin{reseñaplana}
Hoy la geometría sigue siendo el corazón de la ciencia y la ingeniería.  
Es la base del \textbf{diseño arquitectónico}, de la \textbf{construcción de puentes y vehículos}, de la \textbf{óptica} y de la \textbf{física}.  
Pero también es una forma de pensamiento: la capacidad de imaginar, abstraer y representar el espacio.  

Gracias a la geometría analítica (Descartes) y diferencial (Gauss, Riemann), entendemos la curvatura 
de los planetas, el movimiento de los cuerpos y la estructura del espacio-tiempo.  
Incluso el arte moderno, desde el Renacimiento hasta las impresiones digitales, ha sido guiado por 
principios geométricos de perspectiva, simetría y proporción.  

\subsection*{De los planos a los píxeles.}
En el siglo XXI, la geometría vive en los algoritmos y las pantallas.  
Cada imagen, animación o modelo 3D está construido con coordenadas, vectores y polígonos: \textbf{geometría computacional}.  
Los sistemas de navegación GPS usan principios geométricos para determinar posiciones, y los satélites calculan 
rutas en función de triángulos y distancias.  

En la \textbf{inteligencia artificial}, la geometría es el lenguaje oculto que permite a las máquinas “ver”: 
los algoritmos de visión artificial analizan formas, contornos y perspectivas.  
En la \textbf{realidad virtual}, la geometría define los mundos que exploramos; en la \textbf{robótica}, guía 
los movimientos precisos de brazos y sensores.  

De los triángulos grabados en piedra a los polígonos renderizados en pantallas, la geometría sigue 
siendo el arte de comprender el espacio y transformar la imaginación en precisión. 

Una ciencia antigua que no envejece: la que enseña a pensar con líneas, ángulos y belleza.
\end{reseñaplana}

\section*{¿Qué vas a aprender?}
\begin{aprende}
  \item Comprender 
\end{aprende}

\vspace{1cm}
\begin{flushright}
  {\oneptup\itshape ``La geometría brilla, única y eterna, en la mente de Dios; la parte que al hombre le es concedida es una de las razones por las que el hombre es imagen de Dios.''}\\
  {\oneptup - Johannes Kepler}
\end{flushright}



% !TEX root = ../../main.tex
% =============================
% Intro del capítulo
% =============================

% --- Subtítulo ---
\vspace{0.5cm}
\subtitulocapitulo{La ecuación que cambió la forma de resolver el mundo.}

\vspace{1em}
% --- Reseñas históricas ---
\section*{La utilidad de la ecuación cuadrática: una historia que viene de lejos}

\begin{reseñaplana}
Desde hace miles de años, las \textbf{ecuaciones cuadráticas} han sido una herramienta esencial 
para resolver problemas reales: calcular áreas, medir trayectorias o incluso planificar construcciones.  

Mucho antes de que existieran los símbolos del álgebra moderna, los \textbf{babilonios} 
(hacia el siglo XVIII a.C.) ya resolvían ecuaciones de segundo grado usando \textit{figuras geométricas}: 
dividían y completaban cuadrados sobre tablillas de arcilla para hallar longitudes desconocidas. 
Sin notarlo, estaban aplicando los mismos pasos que hoy usamos al ``completar el cuadrado''.  

Siglos después, en la India, el matemático \textbf{Brahmagupta} (siglo VII) dio un paso más al escribir 
reglas que son prácticamente \textit{idénticas a nuestra fórmula cuadrática actual}. Sus métodos permitían 
obtener resultados precisos sin recurrir a geometría, solo con razonamiento numérico. Fue una revolución en su tiempo.
\end{reseñaplana}

\subsection*{Una anécdota que dio nombre al álgebra}

\begin{reseñaplana}
En el siglo IX, el sabio persa \textbf{Al-Khwarizmi} escribió un libro llamado \textit{Al-Jabr}, donde explicaba 
cómo resolver ecuaciones cuadráticas, pero sin usar símbolos ni letras.  Todo lo hacía con palabras y 
diagramas, guiando al lector paso a paso.  

De ese libro surgió el término \textbf{``álgebra''}, derivado de \textit{al-jabr}, que significa 
``restaurar'' o ``recomponer''.  La idea era precisamente eso: \textit{restaurar la igualdad} moviendo 
términos de un lado al otro.  

Con el paso del tiempo, los matemáticos europeos del \textbf{Renacimiento} tomaron esas ideas, 
las expresaron con símbolos y fórmulas, y dieron origen a la \textbf{ecuación cuadrática moderna} que usamos hoy.  
Gracias a ellos, los cálculos se volvieron más rápidos, más sistemáticos… y más bellos.
\end{reseñaplana}

\section*{La ecuación cuadrática en el mundo moderno}

\begin{reseñaplana}
Aunque nació hace más de cuatro mil años, la \textbf{ecuación cuadrática} sigue siendo una de las fórmulas 
más utilizadas del planeta.  
Cada vez que un ingeniero calcula la \textit{trayectoria de un proyectil}, un programador diseña 
la \textit{parábola de un salto en un videojuego}, o un arquitecto dibuja un \textit{arco o una cúpula}, 
está aplicando los mismos principios que descubrieron los babilonios y refinaron los matemáticos del Renacimiento.  

En el siglo XXI, la ecuación cuadrática también vive dentro de las \textbf{máquinas inteligentes}.

Los algoritmos que corrigen la lente de una cámara, los que trazan trayectorias parabólicas en 
simulaciones 3D o incluso los que ajustan modelos de \textit{aprendizaje automático (machine learning)}, 
dependen de cálculos donde la relación entre variables sigue una curva cuadrática.  

Detrás de muchos avances tecnológicos hay, silenciosamente, una vieja amiga del álgebra:  
\textit{la ecuación que relaciona el movimiento, la forma y la simetría.}  
Una fórmula tan simple que cabe en una línea… y tan poderosa que sigue describiendo el mundo cuatro milenios después.
\end{reseñaplana}

\section*{¿Tiene importancia en el mundo moderno? \ldots{} \textbf{¡No poca!}}

\begin{importancialist}
    \item \textbf{Física:} se utiliza para describir movimientos parabólicos, caídas libres y trayectorias de proyectiles, esenciales en la mecánica y la exploración espacial.  
    \item \textbf{Ingeniería:} permite calcular estructuras seguras y eficientes, optimizando fuerzas, curvaturas y ángulos en puentes, túneles y sistemas de energía solar.  
    \item \textbf{Arquitectura:} se aplica en el diseño de arcos, cúpulas y superficies curvas, donde la forma parabólica garantiza estabilidad y belleza estructural.  
    \item \textbf{Informática e Inteligencia Artificial:} interviene en algoritmos de optimización y aprendizaje automático, donde las funciones cuadráticas modelan errores y ajustes en los datos.  
    \item \textbf{Economía y Finanzas:} se usa para analizar costos, beneficios y rendimientos, ayudando a encontrar puntos de equilibrio y maximización de ganancias.  
    \item \textbf{Biología y Medicina:} modela fenómenos naturales como el crecimiento celular, la propagación de epidemias o la dosificación óptima de medicamentos.  
    \item \textbf{Arte, Animación y Diseño 3D:} permite crear curvas suaves y proporciones realistas en gráficos digitales, tipografía y modelado computarizado.  
    \item \textbf{Astronomía:} ayuda a calcular órbitas y trayectorias de cuerpos celestes que siguen patrones parabólicos o hiperbólicos.  
\end{importancialist}

\section*{¿Qué vas a aprender?}
\begin{aprende}
  \item Reconocer la estructura general de una ecuación de segundo grado y sus elementos.  
  \item Resolver ecuaciones cuadráticas usando distintos métodos según los valores de $a$, $b$ y $c$.  
  \item Aplicar el método de factorización y la fórmula general para encontrar las raíces.  
  \item Aprender a completar cuadrados como alternativa para resolver ecuaciones.  
  \item Analizar los diferentes tipos de raíces: reales, múltiples e irracionales.  
  \item Comprender la relación entre los coeficientes y las raíces de una ecuación.  
  \item Traducir problemas cotidianos en ecuaciones cuadráticas y resolverlos.  
\end{aprende}

% --- Cita célebre ---
\vspace{1cm}
\begin{flushright}
  {\oneptup\itshape ``Resolver ecuaciones es domesticar lo desconocido.''}\\
  {\oneptup — Al-Khwarizmi (siglo IX)}
\end{flushright}



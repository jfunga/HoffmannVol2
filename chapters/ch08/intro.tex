% !TEX root = ../../main.tex
% =============================
% Intro del capítulo
% =============================

% --- Subtítulo ---
\vspace{0.5cm}
\subtitulocapitulo{Entre positivos y negativos, todos están a una distancia justa.}

\vspace{1em}
% --- Reseñas históricas ---
\section*{Cuando la distancia empezó a importar.}

\begin{reseñaplana}
La idea del \textbf{valor absoluto} nació mucho antes de que existiera el símbolo que usamos hoy.

En la antigüedad, los matemáticos ya entendían que una distancia no puede ser negativa.  

Los egipcios y babilonios medían terrenos, longitudes o pesos usando siempre cantidades positivas: 
si el cálculo salía “por debajo de cero”, sabían que había un error, pero aún no tenían una forma formal de representarlo.  

Con el paso del tiempo, los griegos empezaron a trabajar con segmentos de recta, donde solo 
interesaba su longitud, no la dirección. Esa noción —la distancia entre dos puntos sin importar el sentido— fue el germen del valor 
absoluto: una medida de “cuánto hay entre A y B”, sin preocuparse por si se avanza hacia la izquierda o hacia la derecha.

\textbf{De los números negativos al símbolo |x|}  
Durante la Edad Media, los comerciantes árabes y los matemáticos indios introdujeron los \textbf{números negativos} en los cálculos.  

Esto trajo un nuevo desafío: ¿cómo expresar la magnitud de una deuda o una pérdida sin usar signos?  

La respuesta llegó lentamente, con la idea de que todo número tiene una “distancia al cero”, positiva o no.  

El símbolo moderno del valor absoluto, las \textbf{barras verticales |x|}, apareció hacia el siglo XIX 
con el matemático alemán \textbf{Karl Weierstrass}.  

Él definió de manera rigurosa lo que ya muchos intuían:  
\[
|x| = 
\begin{cases} 
x, & \text{si } x \ge 0,\\
-x, & \text{si } x < 0.
\end{cases}
\]
Esa simple definición formalizó una idea muy humana: los números negativos no son menos reales, solo miden en dirección contraria.  
\end{reseñaplana}

\section*{Presente: más que distancia, una herramienta de análisis}

\begin{reseñaplana}
Hoy, el valor absoluto es mucho más que una distancia al cero: es una herramienta que mide \textbf{magnitud, error o desviación}.  

En álgebra, nos ayuda a resolver ecuaciones donde una cantidad puede ser positiva o negativa.  

En geometría, permite calcular distancias entre puntos en el plano.  

Y en física, expresa la \textbf{intensidad de fuerzas}, velocidades o desplazamientos sin importar su dirección.  

En la vida cotidiana, el valor absoluto está detrás de muchas decisiones matemáticas simples: cuánto 
te falta para llegar a un punto, qué tan lejos estás de un resultado o cuánto se ha desviado una medida de su valor ideal.  
\end{reseñaplana}

\section*{El valor absoluto en la era digital}  
En el siglo XXI, el concepto de distancia sin signo se volvió esencial en la tecnología.  

\begin{reseñaplana}
Los algoritmos que mueven \textbf{robots y vehículos autónomos} usan el valor absoluto para calcular 
rutas mínimas y márgenes de error.  

Los sistemas de \textbf{inteligencia artificial} lo aplican al ajustar parámetros en modelos de aprendizaje 
automático (a través de funciones de pérdida basadas en |error|).  

En \textbf{informática y videojuegos}, se usa para medir diferencias de posición y determinar colisiones o trayectorias.  

Incluso en la \textbf{economía digital}, las variaciones absolutas sirven para medir riesgos y fluctuaciones en los mercados.  

Así, una idea tan sencilla como medir “la distancia al cero” sigue siendo una de las herramientas más 
universales del pensamiento matemático.  

El valor absoluto nos recuerda que en el mundo —como en las matemáticas— lo importante no siempre es 
el signo, sino la magnitud de lo que sucede.
\end{reseñaplana}


% --- Qué vas a aprender ---
\section*{¿Qué vas a aprender?}
\begin{aprende}
  \item Comprender qué representa el valor absoluto y cómo afecta a una ecuación o inecuación.  
  \item Resolver ecuaciones donde la incógnita aparece dentro de una expresión de valor absoluto.  
  \item Analizar casos en los que la incógnita está tanto dentro como fuera del valor absoluto.  
  \item Aplicar las propiedades del valor absoluto para resolver inecuaciones y representar sus soluciones en la recta real.  
  \item Interpretar el significado geométrico del valor absoluto como distancia y sus aplicaciones en problemas reales.  
\end{aprende}

% --- Cita célebre ---
\vspace{1cm}
\begin{flushright}
  {\oneptup\itshape ``La desigualdad es la esencia de la naturaleza: nada es exactamente igual a otra cosa.''}\\
  {\oneptup — Aristóteles}
\end{flushright}


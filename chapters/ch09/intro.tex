% !TEX root = ../../main.tex
% =============================
% Intro del capítulo
% =============================

% --- Subtítulo ---
\vspace{0.5cm}
\subtitulocapitulo{Donde el álgebra se encuentra con la geometría.}

\vspace{1em}

\section*{De los trazos a los números.}

\begin{reseñaplana}
Durante miles de años, la geometría fue una ciencia de figuras, no de ecuaciones.  
Los \textbf{egipcios} y \textbf{babilonios} trazaban triángulos y cuadriláteros para medir tierras, 
pero no usaban coordenadas ni fórmulas.  
Esa unión llegaría muchos siglos después, cuando el álgebra y la geometría —dos mundos separados— finalmente se encontraron.
El punto de partida fue el trabajo del matemático francés \textbf{René Descartes}, quien en 1637 
publicó su obra \textit{La Géométrie}.  
Allí presentó una idea revolucionaria: usar \textbf{números} para describir \textbf{posiciones} en un plano.  
Cada punto podía definirse mediante un par de valores $(x, y)$, y cada ecuación podía representarse como una curva.  
Así nació la \textbf{geometría analítica}, el puente entre las formas y los números, entre el dibujo y el pensamiento algebraico.  
\end{reseñaplana}

\subsection*{La expansión del plano cartesiano.}
\begin{reseñaplana}
Después de Descartes, el plano se convirtió en una herramienta fundamental para la ciencia.  
El matemático \textbf{Pierre de Fermat} desarrolló en paralelo ideas similares y ayudó a cimentar el concepto de coordenadas.  
Luego, en los siglos XVIII y XIX, matemáticos como \textbf{Leibniz}, \textbf{Euler} y \textbf{Gauss} llevaron esta idea más lejos, 
aplicándola a la física, la astronomía y la ingeniería.  

El plano real permitió representar trayectorias, ondas y fuerzas con precisión, convirtiendo los dibujos de los antiguos en modelos 
matemáticos capaces de predecir el movimiento de los planetas o el comportamiento de los cuerpos en caída libre.  

A medida que la ciencia crecía, el plano también se expandía: del plano real $(x, y)$ al espacio tridimensional $(x, y, z)$, y más 
tarde al plano complejo, donde los números reales se unieron con los imaginarios.  
\end{reseñaplana}


\section*{El lenguaje visual de la matemática moderna.}
\begin{reseñaplana}
Hoy, el plano real es una de las herramientas más universales en la educación y la tecnología.  
Nos permite representar \textbf{funciones, ecuaciones y datos} de forma visual, observar patrones, tendencias y simetrías.  
Cada punto del plano guarda una pareja de valores que puede representar una coordenada geográfica, una medición 
científica o un dato estadístico.  

Los estudiantes usan el plano para entender cómo cambia una variable respecto a otra; los ingenieros lo aplican 
para diseñar estructuras y los científicos lo emplean para estudiar relaciones en la naturaleza.  
El plano real es, en esencia, el mapa donde el pensamiento matemático toma forma.  

\subsection*{De los gráficos a la inteligencia artificial}  
En el siglo XXI, el plano real vive dentro de las pantallas.  
Cada imagen, animación o gráfico digital está formado por millones de puntos con coordenadas $(x, y)$ que se encienden 
o apagan para formar figuras.  
Los programas de diseño, los videojuegos y las simulaciones físicas usan el plano como base para representar movimiento, 
profundidad y perspectiva.  

En la \textbf{inteligencia artificial} y la \textbf{ciencia de datos}, el plano real es el escenario donde los 
algoritmos “aprenden”: cada dato se convierte en un punto, y las líneas o curvas que los conectan son los modelos 
matemáticos que predicen comportamientos.  

Así, el plano real, nacido del trazo de un filósofo en el siglo XVII, sigue siendo el lienzo donde la ciencia moderna dibuja sus ideas.  
Una simple cuadrícula de ejes se transformó en el lenguaje visual del conocimiento humano.
\end{reseñaplana}


% --- Qué vas a aprender ---
\section*{¿Qué vas a aprender?}
\begin{aprende}
  \item Comprender 
\end{aprende}


% --- Cita célebre ---
\vspace{1cm}
\begin{flushright}
  {\oneptup\itshape ``Dar coordenadas a un punto es darle dirección en el infinito.''}\\
  {\oneptup — René Descartes}
\end{flushright}

% !TEX root = ../../main.tex
% =============================
% Intro del capítulo
% =============================

% --- Subtítulo ---
\vspace{0.5cm}
\subtitulocapitulo{Cada exponente, un salto hacia lo infinito}

\vspace{1em} %
% --- Reseñas históricas ---
\section*{Nacimiento: contar sin contar}

\begin{reseñaplana}
Mucho antes de que existiera la palabra “potencia”, los seres humanos ya sabían repetir operaciones.  
Los \textbf{babilonios}, hace más de 4000 años, usaban tablillas de arcilla con cuadrados y cubos de 
los números: eran las primeras \textit{tablas de potencias}.  
Así podían multiplicar rápidamente sin repetir el mismo cálculo una y otra vez.  
Aunque no conocían aún la notación moderna, comprendían que elevar un número era una forma de resumir la multiplicación.  

Los egipcios también tenían su propio método: la \textbf{duplicación}.  
Para multiplicar, sumaban una cantidad consigo misma tantas veces como fuera necesario, una idea que, 
siglos después, daría lugar al exponente. 

\textbf{El nacimiento del exponente}  
En la Edad Media, los matemáticos árabes y europeos ya usaban potencias sin símbolos.  
Escribían frases como “cosa al cuadrado” o “cubo de la cosa”, inspiradas en figuras geométricas:  
“cuadrado” venía de un área, “cubo” de un volumen.  

La notación moderna comenzó en el siglo XVII.  
El francés \textbf{René Descartes}, en su obra \textit{La Géométrie} (1637), introdujo los exponentes 
como los conocemos hoy: $x^2$, $x^3$, etc.  
Esta idea revolucionó el álgebra, porque permitió escribir operaciones repetidas con claridad y 
trabajar con ellas como si fueran objetos manipulables.  

Más tarde, matemáticos como \textbf{Euler} y \textbf{Newton} extendieron el concepto a exponentes 
fraccionarios, negativos e incluso irracionales, mostrando que las potencias podían ir mucho más 
allá de la simple multiplicación.  
\end{reseñaplana}

% --- Importancia actual ---
\section*{El lenguaje del crecimiento y la energía}

\begin{reseñaplana}
Hoy, la potenciación está en todas partes.  
En física, describe cómo crecen la energía, la presión o la velocidad.  
En biología, explica el crecimiento exponencial de poblaciones o la propagación de virus.  
En economía, modela los intereses compuestos, donde el dinero “crece sobre sí mismo”.  
Y en informática, define la rapidez con la que aumentan las capacidades de procesamiento: 
cada nueva generación de chips es una potencia de la anterior.  

El concepto de potencia nos enseña una verdad simple pero 
poderosa: \textit{pequeñas repeticiones pueden generar grandes resultados.}
\end{reseñaplana}

\section*{La era exponencial} 
\begin{reseñaplana}
En la era digital, la potenciación se ha convertido en un símbolo de progreso.  
Las computadoras funcionan a velocidades que crecen exponencialmente, las redes neuronales de la inteligencia 
artificial multiplican parámetros como potencias, y el almacenamiento de datos se mide en teras, 
petas y exas —cada uno una potencia de diez mayor que el anterior.  

Incluso fenómenos sociales, como el crecimiento de redes o la difusión de información, siguen patrones \textbf{exponenciales}.  
Lo que comenzó como una forma de abreviar multiplicaciones hoy explica cómo evoluciona el mundo moderno.  

La potenciación es, en esencia, la matemática del cambio acelerado:  
una idea antigua que nunca ha dejado de crecer.
\end{reseñaplana}

\section*{¿Qué vas a aprender?}
\begin{aprende}
  \item Comprender qué significa elevar un número a una potencia y cómo se representa.  
  \item Aplicar las reglas básicas de las potencias para multiplicar, dividir y combinar expresiones con la misma base.  
  \item Analizar qué ocurre cuando una potencia se eleva a otra o cuando se aplica a un producto o una fracción.  
  \item Entender el papel de los exponentes negativos y cómo transforman una potencia en su recíproco.  
  \item Descubrir el significado del exponente cero y por qué toda potencia con ese exponente vale uno.  
  \item Usar las propiedades de las potencias para simplificar expresiones y resolver problemas algebraicos.  
\end{aprende}

\vspace{1.2cm} % espacio arriba
\begin{flushright}
  {\fontsize{12}{14}\selectfont\itshape
  ``La potencia de un número es el eco de su multiplicación repetida hasta el infinito.''\\[6pt]
  — Hermann Hankel}%
\end{flushright}




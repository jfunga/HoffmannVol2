% !TEX root = ../../../main.tex

% =============================
% CASO 1
% =============================

\begin{Caso}[Cálculo de expresión con potencias de distintos signos]
  \casolinea{Calcular la siguiente expresión:}{(-4)^2 - (-2)^3 + (-3)^3 - (-1)^{10}}

  % Nota del profesor a TODO el ancho y centrada:
  \casoRecuerda{potencias \textit{pares} dan resultado positivo; potencias \textit{impares} conservan el signo.}

  \casolineaWc{Calculamos primero cada potencia (encerramos en corchetes los resultados):}{= [16] - [-8] + [-27] - [1]}
  \casolinea{Eliminamos los signos de agrupación:}{= 16 + 8 - 27 - 1}
  \casolinea{Sumamos los términos:}{= \Resultado{-4}}
\end{Caso}









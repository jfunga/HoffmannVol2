% !TEX root = ../../../main.tex

% =============================
% CASO 1
% =============================

%\begin{Caso}[Cálculo de expresión con potencias de distintos signos]
  %\casolinea{Calcular la siguiente expresión:}{(-4)^2 - (-2)^3 + (-3)^3 - (-1)^{10}}
  %\casolinea{Recuerda que las potencias pares dan positivo, mientras que las impares conservan el signo.}{= [16] - [-8] + [-27] - [1]}
  %\casolinea{Eliminamos los signos de agrupación:}{= 16 + 8 - 27 - 1}
  %\casolinea{Sumando los términos obtenemos:}{\Resultado{-4}}
%\end{Caso}

\begin{Caso}[Cálculo de expresión con potencias de distintos signos (A)]
  \casolinea{Calcular la siguiente expresión:}{(-4)^2 - (-2)^3 + (-3)^3 - (-1)^{10}}

  % NUEVA micro-observación (una línea)
  \casolinea{Recuerda: potencias \textit{pares} dan resultado positivo; potencias \textit{impares} conservan el signo.}{}

  \casolinea{Calculamos primero cada potencia (encerramos en corchetes los resultados):}{= [16] - [-8] + [-27] - [1]}
  \casolinea{Eliminamos los signos de agrupación:}{= 16 + 8 - 27 - 1}

  % CAMBIO: usar \Resultado{-4} en vez de \fbox{$-4$}
  \casolinea{Sumamos los términos:}{= \Resultado{-4}}
\end{Caso}







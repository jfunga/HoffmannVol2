% !TEX root = ../../../main.tex

% =============================
% CASO 3
% =============================

\begin{CasoL}[Cálculo de expresión con potencias de distintos signos]
  % Larga → centrada debajo:
  \casolineaW{Calcular la siguiente expresión:}{
    (-1)^3\left[(-3)^2-(-2)^4\right]
    -(-2)^7\left[(-5)^2-5^2\right]
    +(-1)^4\left[(-3)^3-3^3-(-4)^3\right]
  }

  % Cortas → dos columnas:
  \casolineaW{Calculamos cada una de las potencias que aparecen en la expresión:}{
    =(-1)[9-16]-(-128)[25-25]+(1)[-27-27-(-64)]
  }

  \casolinea{Sumamos los términos internos de cada corchete:}{
    =(-1)(-7)-(-128)\cdot 0+10
  }

  \casolinea{Efectuamos el primer producto (el segundo da cero):}{=7+10}

  \casolinea{Sumamos:}{=\Resultado{17}}
\end{CasoL}








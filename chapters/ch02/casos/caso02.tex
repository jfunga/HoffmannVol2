% !TEX root = ../../../main.tex

% =============================
% CASO 2
% =============================

\begin{Caso}[Cálculo de expresión con potencias de distintos signos]
  \casolinea{Calcular la siguiente expresión:}{(-3)^4-3^4+(-2)^8-2^3-(-5)^3-(-7)^2}

  % NUEVA micro-observación (una línea)
  \casolinea{Recuerda: potencias \textit{pares} dan resultado positivo; potencias \textit{impares} conservan el signo.}{}

  \casolinea{Calculamos cada una de las potencias señaladas. Para evitar confusiones, encerramos en corchetes los resultados:}{=[81]-[81]+[256]-[8]-[-125]-[49]}
  \casolinea{Eliminamos los signos de agrupación:}{=81-81+256-8+125-49}

  % CAMBIO: usar \Resultado{-4} en vez de \fbox{$-4$}
  \casolinea{Sumamos los términos:}{= \Resultado{324}}
\end{Caso}



% !TEX root = ../../main.tex
\begin{capitulobox}
Si tu autoestima está baja, no te preocupes: \textbf{\textit{elévala al cuadrado.}}
\end{capitulobox}

% ==========================
% Capítulo 2 — Potenciación
% ==========================

\subtitulocapitulo{Potencia}

Llamaremos potencia \emph{n}-sima de una cantidad al resultado de
multiplicar \emph{n} veces esa cantidad por sí misma.

Por ejemplo, la quinta potencia de 2, que indicaremos así:

\[
  2^5 \;=\;
  \llaveabajo{2 \cdot 2 \cdot 2 \cdot 2 \cdot 2}{5 veces}
  \;=\; 32
\]

Otros ejemplos:  

\begin{ejemplosplain}[2]
  \task $x^4 = x \cdot x \cdot x \cdot x$
  \task $(-5)^3 = (-5)\cdot(-5)\cdot(-5) = -125$
\end{ejemplosplain}

En una potencia se distinguen dos elementos: la \textit{base}, que es el
número que se multiplica por sí mismo, y el \textit{exponente}, que indica
cuántas veces esto se hace.

En la expresión \(2^n\), la base es 2 y el exponente es \(n\).

Dado que el producto de números positivos da como resultado un número
positivo, toda potencia con base positiva dará, al ser resuelta, una
cantidad positiva.  

En cambio, si la base es un número negativo, hay que distinguir dos
casos:

a) Si el exponente es \textit{impar}, la potencia será negativa: 

\begin{ejemplosplain}[2]
  \task $(-7)^3 = (-7)\cdot(-7)\cdot(-7) = -343$
  \task $(-3)^5 = (-3)\cdot(-3)\cdot(-3)\cdot(-3)\cdot(-3) = -243$
\end{ejemplosplain}

b) Si el exponente es \textit{par}, la potencia será positiva: 

\begin{ejemplosplain}[2]
  \task $(-2)^4 = (-2)\cdot(-2)\cdot(-2)\cdot(-2) = 16$
  \task $(-11)^2 = (-11)\cdot(-11) = 121$
\end{ejemplosplain}

Por convención, cuando el exponente es la unidad, ésta no se indica:

\noindent\makebox[\linewidth][c]{%
  \begin{minipage}{0.25\linewidth} % ajusta 0.6 a tu gusto
    \begin{ejemplosplain}
      \task $8^1 = 8$
      \task $(-4)^1 = -4$
      \task $x^1 = x$
    \end{ejemplosplain}
  \end{minipage}%
}

% !TEX root = ../../../main.tex
% =============================
% EJERCICIO (1)
% =============================
\begin{BloqueEjercicios}[Potencias con exponentes enteros]
  \begin{ej3col}
    \item $2^6$
    \item $(-3)^2$
    \item $(-5)^4$
    \item $(-2)^3$
    \item $7^2$
    \item $(-6)^3$
    \item $0^7$
    \item $(-2)^5$
    \item $(-3)^3$
    \item $(-1)^{33}$
    \item $10^3$
    \item $(-1)^{14}$
  \end{ej3col}
\end{BloqueEjercicios}


Es importante tener en cuenta que, si una potencia va precedida por
algún signo, el resultado final queda afectado por dicho signo.  

Ejemplos:

\begin{ejemplos}[1][\textbullet]
  \task $(-2)^4 = (-2)\cdot(-2)\cdot(-2)\cdot(-2) = 16$
  \task $-(-2)^4 = -[(-2)\cdot(-2)\cdot(-2)\cdot(-2)] = -16$
  \task $-2^4 = -[2 \cdot 2 \cdot 2 \cdot 2] = -16$
  \task $(-3)^3 = (-3)\cdot(-3)\cdot(-3) = -27$
  \task $-(-3)^3 = -[(-3)\cdot(-3)\cdot(-3)] = 27$
  \task $-3^3 = -[3 \cdot 3 \cdot 3] = -27$
  \task $-a^3 = -[a \cdot a \cdot a] = -a^3$
  \task $-(-a)^3 = -[(-a)\cdot(-a)\cdot(-a)] = a^3$
\end{ejemplos}


% Ejercicio 2
% !TEX root = ../../../main.tex
% =============================
% EJERCICIO (2)
% =============================

\begin{BloqueEjercicios}[Signos y potencias con paréntesis]
  \begin{ej3col}
    \item $-(-1)^4$
    \item $-(-1)^5$
    \item $-(-1)^6$
    \item $-(-4)^3$
    \item $-4^2$
    \item $(-4)^2$
    \item $-(-x)^4$
    \item $-8^2$
    \item $(-8)^2$
    \item $-(-8)^2$
  \end{ej3col}
\end{BloqueEjercicios}


Lo anterior se presenta frecuentemente al hacer operaciones más complejas, como puede observarse en los siguientes ejemplos:

% Caso 1 (enseñanzas del profesor)
% !TEX root = ../../../main.tex

% =============================
% CASO 1
% =============================

%\begin{Caso}[Cálculo de expresión con potencias de distintos signos]
  %\casolinea{Calcular la siguiente expresión:}{(-4)^2 - (-2)^3 + (-3)^3 - (-1)^{10}}
  %\casolinea{Recuerda que las potencias pares dan positivo, mientras que las impares conservan el signo.}{= [16] - [-8] + [-27] - [1]}
  %\casolinea{Eliminamos los signos de agrupación:}{= 16 + 8 - 27 - 1}
  %\casolinea{Sumando los términos obtenemos:}{\Resultado{-4}}
%\end{Caso}


% =============================
% CASO 1 - PRUEBA DE A POCO
% =============================

\begin{Caso}[Cálculo de expresión con potencias de distintos signos]
  \casolinea{Calcular la siguiente expresión:}{(-4)^2 - (-2)^3 + (-3)^3 - (-1)^{10}}

  % NUEVA micro-observación (una línea)
  \casolinea{Recuerda: potencias \textit{pares} dan resultado positivo; potencias \textit{impares} conservan el signo.}{}

  \casolinea{Calculamos primero cada potencia (encerramos en corchetes los resultados):}{= [16] - [-8] + [-27] - [1]}
  \casolinea{Eliminamos los signos de agrupación:}{= 16 + 8 - 27 - 1}

  % CAMBIO: usar \Resultado{-4} en vez de \fbox{$-4$}
  \casolinea{Sumamos los términos:}{= \Resultado{-4}}
\end{Caso}








% Caso 2 (enseñanzas del profesor)
% !TEX root = ../../../main.tex

% =============================
% CASO 2
% =============================

\begin{Caso}[Cálculo de expresión con potencias de distintos signos (B)]
  \casolinea{Calcular la siguiente expresión:}{(-3)^4-3^4+(-2)^8-2^3-(-5)^3-(-7)^2}

  \casolinea{Calculamos cada una de las potencias señaladas. Para evitar \\ 
  confusiones, encerramos en corchetes los resultados:}{=[81]-[81]+[256]-[8]-[-125]-[49]}
  \casolinea{Eliminamos los signos de agrupación:}{=81-81+256-8+125-49}

  % CAMBIO: usar \Resultado{-4} en vez de \fbox{$-4$}
  \casolinea{Sumamos los términos:}{= \Resultado{324}}
\end{Caso}

% Caso 3 (enseñanzas del profesor)
% !TEX root = ../../../main.tex

% =============================
% CASO 3
% =============================

\begin{CasoL}[Cálculo de expresión con potencias de distintos signos]
  \casolineaWc{Calcular la siguiente expresión:}{(-1)^3\left[(-3)^2-(-2)^4\right]-(-2)^7\left[(-5)^2-5^2\right]+(-1)^4\left[(-3)^3-3^3-(-4)^3\right]}
  \casolineaWc{Calculamos cada una de las potencias que aparecen en la expresión:}{=(-1)[9-16]-(-128)[25-25]+(1)[-27-27-(-64)]}

  \casolineaWc{Sumamos los términos internos de cada corchete:}{=(-1)(-7)-(-128)\cdot 0+10}
  \casolineaWc{Efectuamos el primer producto (el segundo da cero):}{=7+10}

  \casolinea{Sumamos:}{=\Resultado{17}}
\end{CasoL}










% Ejercicio 3

\begin{BloqueEjercicios}[Calcular el valor de cada una de las siguientes expreciones:]
  \begin{ej2col}
    \item $5^3+(-5)^3$ % (3)
    \item $(-2)^3-2^3$ % (1)
    \item $5^3+(-5)^3+(-3)^2-3^2$ % (2)
    \item $(-2)^5-(-2)^4+(-3)^3-2^4$ % (5)
    \item $(-2)^6 - (-2)^5+(-3)^4-2^5+(-4)^3$
    \item $(-4)^2-(-2)^3-(-7)^2+(-5)^2$ % (7)
    \item $(-1)^5(-2)^5+(-1)^{10}(-3)^4-(-1)^6(-8)^2$ % (8)
    \item $a^5-(-a)^5-a^4+(-a)^4$ % (4)
    \item $x^3+(-x)^4-(-x)^3+x^4$ % (6)
    \item $(-2)^3\left[2^3-(-2)^3\right]+(-2)^4\left[(-2)^4-2^4\right]$ % (9)
  \end{ej2col}

  % #11 centrado en su propia línea
  \vspace{.6\baselineskip}
  \noindent\makebox[\linewidth][c]{{11.}\; 
    $(-2)^2\left[(-2)^5+(-2)^4-(-2)^3\right]-(-2)^3\left[(-2)^6-2^5-(-2)^4\right]$}
  \end{BloqueEjercicios}





\section*{Producto de potencias de igual base}

El resultado de multiplicar potencias de igual base es esa base elevada a la suma de los exponentes de los factores:

% Regla en caja con numeración (1)

\ReglaCajaTag{x^{m}\cdot x^{n}=x^{m+n}}{1}

\noindent\textbf{En efecto:}
\vspace*{-0.05cm}
\begingroup
\renewcommand{\arraystretch}{1.75}% ← separa filas (1.25–1.45 suele verse bien)
\[
\begin{array}{@{}l rcl@{}}
\makebox[0pt][r]{\text{a-. }\;} & 2^{3}\!\cdot\! 2^{4} & = &
  \underbrace{(2\!\cdot\!2\!\cdot\!2)}_{\textit{3 veces}}
  \cdot \underbrace{(2\!\cdot\!2\!\cdot\!2\!\cdot\!2)}_{\textit{4 veces}}
  = \underbrace{2\!\cdot\!2\!\cdot\!2\!\cdot\!2\!\cdot\!2\!\cdot\!2\!\cdot\!2}_{\textit{7 veces}}
  = 2^{7} \\

\makebox[0pt][r]{\text{b-. }\;} & a^{5}\!\cdot\! a^{3} & = &
  \underbrace{(a\!\cdot\!a\!\cdot\!a\!\cdot\!a\!\cdot\!a)}_{\textit{5 veces}}
  \cdot \underbrace{(a\!\cdot\!a\!\cdot\!a)}_{\textit{3 veces}}
  = \underbrace{a\!\cdot\!a\!\cdot\!a\!\cdot\!a\!\cdot\!a\!\cdot\!a\!\cdot\!a\!\cdot\!a}_{\textit{8 veces}}
  = a^{8} \\

\makebox[0pt][r]{\text{c-. }\;} & x^{2}\!\cdot\! x^{4}\!\cdot\! x\!\cdot\! x^{3} & = &
  \underbrace{(x\!\cdot\!x)}_{\textit{2 veces}}
  \cdot \underbrace{(x\!\cdot\!x\!\cdot\!x\!\cdot\!x)}_{\textit{4 veces}}
  \cdot x \cdot \underbrace{(x\!\cdot\!x\!\cdot\!x)}_{\textit{3 veces}} \\[.2ex]
& & = & \underbrace{x\!\cdot\!x\!\cdot\!x\!\cdot\!x\!\cdot\!x\!\cdot\!x\!\cdot\!x\!\cdot\!x\!\cdot\!x\!\cdot\!x}_{\textit{10 veces}}
  = x^{10}
\end{array}
\]
\endgroup

% Caso 4 
% !TEX root = ../../../main.tex

% =============================
% CASO 4
% =============================

\begin{Caso}[Producto de potencias]
  \casolinea{Multiplicar:}{48\,a^5 b^2 \cdot 54\,a^2 b c^2}

  \casoRecuerda{Trabaja con orden y por etapas.}

  \casolinea{Descomponemos los coeficientes en factores primos:}{= 2^4 \cdot 3 \cdot a^5 \cdot b^2 \cdot 2 \cdot 3^3 \cdot a^2 \cdot b \cdot c^2}
  \casolinea{Multiplicamos los factores con igual base sumando sus exponentes:}{= 2^{4+1} \cdot 3^{1+3} \cdot a^{5+2} \cdot b^{2+1} \cdot c^2}
  \casolinea{Resolvemos:}{= \Resultado{2^5 \cdot 3^4 \, a^7 b^3 c^2}}
\end{Caso}







% Caso 5
% !TEX root = ../../../main.tex

% =============================
% CASO 5
% =============================

% Caso 6
% !TEX root = ../../../main.tex

% =============================
% CASO 6
% =============================

\textbf{Nota:} El signo del resultado final pudo determinarse desde el primer momento tomando en cuenta que en 
la expresión original hay ocho factores negativos (3 que provienen del segundo multiplicando, 4 que provienen del 
tercero y otro más del cuarto). Emplearemos esta forma en el ejercicio que sigue.

% Caso 7
% !TEX root = ../../../main.tex

% =============================
% CASO 7
% =============================


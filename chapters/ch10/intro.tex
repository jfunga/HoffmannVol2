% !TEX root = ../../main.tex
% =============================
% Intro del capítulo
% =============================

% --- Subtítulo ---
\vspace{0.5cm}
\subtitulocapitulo{Una fórmula con propósito: transformar y conectar.}

\vspace{1em}

\section*{Cuando los números empezaron a depender unos de otros.}

\begin{reseñaplana}
Desde los primeros tiempos, los matemáticos buscaron entender cómo cambiaba una cantidad cuando otra variaba.  
Los \textbf{babilonios} ya registraban tablas donde a cada valor de entrada le correspondía un resultado, 
por ejemplo, para calcular áreas o volúmenes.  
Aunque no hablaban de “funciones”, estaban construyendo sus primeras tablas de correspondencia: el origen de la idea.  

Siglos después, los \textbf{griegos} estudiaron relaciones entre longitudes y ángulos, y los astrónomos indios 
y árabes elaboraron tablas trigonométricas para predecir eclipses o posiciones de planetas.  
Esa intuición —que un número puede depender de otro— fue la semilla del concepto de función.  
\end{reseñaplana}

\section*{Del pensamiento geométrico a la fórmula.}
\begin{reseñaplana}
Durante el Renacimiento y los siglos XVII y XVIII, la función adquirió forma y nombre.  
El alemán \textbf{Gottfried Leibniz} usó por primera vez el término “función” hacia 1694 para describir cualquier 
magnitud relacionada con otra.  
Poco después, \textbf{Euler} la definió de manera formal como una “regla que asigna a cada número $x$ un único 
valor $y$”, y escribió por primera vez la notación moderna $f(x)$.  

Gracias a esta simple idea, las matemáticas pudieron describir el movimiento, el crecimiento y las leyes de la naturaleza con fórmulas.  
Las funciones se convirtieron en el lenguaje del cambio, y sin ellas no existiría el cálculo ni la física moderna.  
\end{reseñaplana}

\section*{De las fórmulas a los gráficos.}
\begin{reseñaplana}
Hoy, el plano real es una de las herramientas más universales en la educación y la tecnología.  Hablar de funciones es 
hablar de relaciones, patrones y modelos.  
Una función no es solo una fórmula: también puede representarse con un gráfico, una tabla o una regla verbal.  
Los estudiantes las usan para estudiar variaciones, máximos, mínimos y tendencias, mientras los científicos las 
aplican para modelar fenómenos del mundo real.  

En la vida diaria, encontramos funciones por todas partes:  
la temperatura cambia con la hora, el costo depende del consumo, la velocidad varía con el tiempo.  
Cada situación donde una variable influye en otra es una función en acción.
\end{reseñaplana}  

\section*{El corazón de la tecnología inteligente} 
\begin{reseñaplana} 
En el siglo XXI, las funciones se convirtieron en el motor silencioso de la tecnología.  
En la \textbf{inteligencia artificial}, las funciones matemáticas modelan las conexiones entre neuronas artificiales.  
En la \textbf{informática}, transforman datos de entrada en resultados; en la \textbf{física}, describen el 
comportamiento de ondas, campos y partículas; y en la \textbf{economía}, explican cómo las decisiones afectan los mercados.  

Cada vez que un algoritmo predice una imagen, una voz o una tendencia, detrás hay una función trabajando.  
Hoy, las funciones son más que un tema escolar: son el lenguaje con el que las máquinas y los científicos entienden el mundo.  

En resumen, una función es una historia de dependencia:  
\textit{cómo un número, una variable o una idea puede transformar otra}.
\end{reseñaplana}

\section*{¿Qué vas a aprender?}
\begin{aprende}
  \item Comprender 
\end{aprende}


\vspace{1cm}
\begin{flushright}
  {\oneptup\itshape ``Las funciones son la música de las matemáticas, donde una nota responde siempre a otra.''}\\
  {\oneptup — Richard Courant}
\end{flushright}



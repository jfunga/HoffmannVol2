% !TEX root = ../../main.tex
% =============================
% Intro del capítulo
% =============================

% --- Subtítulo ---
\vspace{0.5cm}
\subtitulocapitulo{De puntos a regiones: la geometría de la desigualdad.}

\vspace{1em}

\section*{Antes de los signos: vivir con límites .}

\begin{reseñaplana}
 Mucho antes de que existieran los símbolos ``$<$'' y ``$>$'', los antiguos ya pensaban en desigualdades.  
Los \textbf{babilonios} y \textbf{egipcios} comparaban cantidades al medir terrenos, intercambiar mercancías 
o construir templos, usando aproximaciones y márgenes de error.  
En la \textbf{Grecia clásica}, matemáticos como \textbf{Eudoxo} y \textbf{Euclides} desarrollaron la idea de 
comparar magnitudes dentro de la teoría de razones y proporciones: saber cuándo una cantidad es mayor, 
igual o menor que otra, incluso sin usar decimales.  

El ejemplo más famoso llega con \textbf{Arquímedes}, quien no dio un valor exacto para $\pi$, sino que lo \textit{acotó} entre dos números:
\[
\frac{223}{71} < \pi < \frac{22}{7}.
\]
Este simple intervalo es una de las primeras inecuaciones registradas en la historia: una forma precisa de 
decir “sabemos dónde está, aunque no sepamos exactamente cuánto vale”.  

Mientras tanto, en la antigua \textbf{China}, el texto \textit{Los Nueve Capítulos sobre el Arte Matemático} (siglos I a.C.–I d.C.) 
enseñaba métodos para resolver problemas reales de obras públicas y comercio, siempre bajo condiciones como longitudes positivas o cantidades 
limitadas: los primeros pasos del pensamiento con restricciones. 
\end{reseñaplana}

\section*{Nace la notación: del “es mayor que” al “$>$”}

\begin{reseñaplana}
Durante siglos, las desigualdades se escribían con palabras.  
Todo cambió con la llegada del \textbf{simbolismo moderno}.  
Después de que \textbf{Robert Recorde} introdujera el signo igual (=) en 1557, el matemático inglés \textbf{Thomas Harriot} 
publicó en 1631 los símbolos ``$<$'' y ``$>$'' en su obra \textit{Artis Analyticae Praxis}.  
Por primera vez, las comparaciones numéricas se podían expresar de forma breve y universal.  

Los símbolos ``$\leq$'' y ``$\geq$'' aparecieron un poco más tarde.  
El inglés \textbf{John Wallis} propuso una versión en 1670, y el francés \textbf{Pierre Bouguer} les dio su forma moderna en 1734.  
Desde entonces, el lenguaje simbólico de las inecuaciones quedó prácticamente establecido.  

\textbf{Del taller al pizarrón: cuando las desigualdades se vuelven ciencia}  
Durante los siglos XIX y XX, las inecuaciones dejaron de ser simples comparaciones y se convirtieron en herramientas poderosas.  
Matemáticos como \textbf{Cauchy}, \textbf{Schwarz}, \textbf{Jensen} y \textbf{Chebyshev} desarrollaron desigualdades fundamentales 
que hoy se usan en física, análisis, estadística y teoría de la información.  

- La \textbf{desigualdad de Cauchy-–Schwarz} define límites entre productos y magnitudes.  
- La \textbf{media aritmética–-geométrica (AM-–GM)} establece comparaciones entre promedios.  
- La \textbf{desigualdad de Chebyshev} permite estimar probabilidades sin conocer la forma exacta de una distribución.  
- La \textbf{desigualdad de Jensen} explica cómo se comportan los promedios de funciones curvas (convexas).  

Estas ideas transformaron las desigualdades en una parte esencial de la matemática moderna.  

\textbf{La era de optimizar con restricciones}  
En el siglo XX, las inecuaciones se convirtieron en lenguaje de decisiones.  
Cuando escribimos un sistema del tipo
\[
Ax \leq b,
\]
estamos definiendo todas las soluciones posibles dentro de un espacio: un \textbf{poliedro}.  
Resolver problemas ahora significaba elegir el mejor punto dentro de ese conjunto.

En 1947, \textbf{George Dantzig} creó la \textbf{programación lineal} y el famoso \textbf{método símplex}, revolucionando la forma 
en que las industrias planifican producción, logística, finanzas o redes.  
Las inecuaciones pasaron del papel al mundo real: calcular costos, optimizar recursos, maximizar beneficios.  
\end{reseñaplana}

\section*{Las inecuaciones en la era digital (2025)}

\begin{reseñaplana}  
Las desigualdades siguen marcando los límites del mundo, pero ahora lo hacen en los circuitos y algoritmos.  
Se usan en \textbf{inteligencia artificial} para ajustar redes neuronales, en \textbf{criptografía} para mantener la seguridad 
de los datos y en \textbf{computación cuántica} para definir regiones de probabilidad.  
Los sistemas de inecuaciones ayudan a que los satélites corrijan trayectorias, los autos autónomos calculen distancias 
seguras y los teléfonos optimicen energía y señal.

Más de dos mil años después, las inecuaciones siguen cumpliendo el mismo propósito:  
\textit{poner límites, proteger el equilibrio y ayudarnos a decidir con lógica en un mundo lleno de posibilidades.}
\end{reseñaplana}

\section*{¿Qué vas a aprender?}
\begin{aprende}
  \item Comprender la recta real como el espacio donde se representan números y soluciones de inecuaciones.  
  \item Reconocer las relaciones de orden y cómo dan origen a desigualdades e inecuaciones.  
  \item Aplicar las propiedades básicas de las desigualdades para transformar y resolver expresiones algebraicas.  
  \item Resolver inecuaciones lineales con una o más incógnitas y analizar sus posibles soluciones.  
  \item Representar las soluciones en la recta real o mediante intervalos de forma clara y ordenada.  
  \item Resolver sistemas de inecuaciones y comprender las regiones que cumplen simultáneamente las condiciones.  
\end{aprende}

% --- Cita célebre ---
\vspace{1cm}
\begin{flushright}
  {\oneptup\itshape ``La desigualdad es la esencia de la naturaleza: nada es exactamente igual a otra cosa.''}\\
  {\oneptup — Aristóteles}
\end{flushright}


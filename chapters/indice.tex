% !TEX root = ../main.tex
\chapter*{ÍNDICE}
\markboth{ÍNDICE}{ÍNDICE}

% ===== columnas (1: Nº, 2: signo, 3: título) =====
\newlength{\NumW}     \setlength{\NumW}{0.90cm}   % Nº
\newlength{\SignW}    \setlength{\SignW}{1.20cm}  % signo
\newlength{\GapSmall} \setlength{\GapSmall}{0.50em} % espacio pequeño sig→título

% Fila suelta (una tabla por fila ⇒ LaTeX puede cortar entre filas)
\newcommand{\IndiceFila}[3]{%
  \noindent\textbf{CAPÍTULO #1}\par
  \noindent\begin{tabularx}{\textwidth}{@{} >{\raggedleft\arraybackslash}p{\SignW} @{\hspace{\GapSmall}} X @{}}
    {\large\boldmath$#2$} & \textbf{#3}
  \end{tabularx}\par\vspace{-\parskip}
}

% Subtítulos bajo el inicio del título (col. 3)
\newlength{\BeforeTitle}
\newlength{\SubIndent}\setlength{\SubIndent}{0.6em}
\setlength{\BeforeTitle}{0pt}
\addtolength{\BeforeTitle}{\SignW}
\addtolength{\BeforeTitle}{\GapSmall}
\addtolength{\BeforeTitle}{\SubIndent}
\newcommand{\IndiceSubtitulos}[1]{%
  \noindent\makebox[\BeforeTitle][l]{}%
  \begin{minipage}[t]{\dimexpr\textwidth-\BeforeTitle\relax}
    
    \begin{itemize}[label={}, labelsep=0pt, leftmargin=0pt, itemsep=0.2em, topsep=0.1em]
      #1
    \end{itemize}
  \end{minipage}\par
}

% =========================
% PARTE I
% =========================
\begin{center}{\Large\bfseries Parte I — El universo de los números y sus propiedades}\end{center}
\begin{center}{\vspace{-0.4em}\Large\itshape Fundamentos del pensamiento algebraico}\end{center}

\IndiceFila{1}{\Large\pi}{NÚMEROS IRRACIONALES}
\IndiceSubtitulos{
  \item RAZÓN O RELACIÓN
  \item NÚMEROS RACIONALES
  \item NÚMEROS IRRACIONALES
  \item Historia del número $\pi$
  \item NÚMEROS REALES
}

\IndiceFila{2}{a^n}{POTENCIACIÓN}
\IndiceSubtitulos{
  \item POTENCIA
  \item Producto de potencias de igual base
  \item Potencia de potencia
  \item Potencia de un producto
  \item Cociente de potencias de igual base
  \item Exponentes negativos (1)
  \item Exponentes negativos (2)
  \item Potencia de una fracción
  \item Exponentes negativos 
  \item El exponente cero
}

\IndiceFila{3}{\sqrt[\leftroot{-2}\uproot{4}n]{\ }}{RADICACIÓN}
\IndiceFila{4}{\tfrac{1}{\sqrt{\ }}}{RACIONALIZACIÓN}

% =========================
% PARTE II
% =========================
\begin{center}{\Large\bfseries Parte II — Las ecuaciones y su lenguaje}\end{center}
\begin{center}{\vspace{-0.4em}\Large\itshape De lo lineal a lo cuadrático}\end{center}

\IndiceFila{5}{x^2}{ECUACIÓN DE SEGUNDO GRADO}
\IndiceFila{6}{\{\}}{SISTEMAS DE ECUACIONES}
\IndiceFila{7}{\geq}{INECUACIONES}
\IndiceFila{8}{|x|}{VALOR ABSOLUTO}

% =========================
% PARTE III
% =========================
\begin{center}{\Large\bfseries Parte III — El plano y la forma}\end{center}
\begin{center}{\vspace{-0.4em}\Large\itshape Del álgebra a la geometría}\end{center}

\IndiceFila{9}{\mathbb{R}^2}{EL PLANO REAL}
\IndiceFila{10}{f(x)}{FUNCIONES}
\IndiceFila{11}{\leq}{INECUACIONES CUADRÁTICAS}
\IndiceFila{12}{\mathbb{D}}{ECUACIONES DIOFÁNTICAS}
\IndiceFila{13}{\triangle}{GEOMETRÍA (Pitágoras, Tales, Euclides)}
\IndiceFila{14}{\infty}{\textbf{Problemas de recapitulación}} % negritas, no mayúsculas

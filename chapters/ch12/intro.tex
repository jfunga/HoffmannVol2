% !TEX root = ../../main.tex% !TEX root = ../../main.tex
% =============================
% Intro del capítulo
% =============================

% --- Subtítulo ---
\vspace{0.5cm}
\subtitulocapitulo{Buscando soluciones enteras a problemas eternos.}

\vspace{1em}
% --- Reseñas históricas ---
\section*{Números enteros en la antigüedad.}

\begin{reseñaplana}
Desde los primeros registros, los humanos se interesaron por los números enteros y las proporciones exactas.  
Los egipcios y babilonios ya resolvían problemas que hoy llamaríamos \textbf{diofánticos}: encontrar números 
que cumplan condiciones específicas, como “tres longitudes que formen un triángulo rectángulo perfecto”.  
Ejemplos como $(3,4,5)$ o $(5,12,13)$ —las famosas ternas pitagóricas— son las primeras ecuaciones diofánticas conocidas: 
relaciones entre enteros que encajan perfectamente.  
Pero el verdadero nacimiento del tema llegó siglos después, con un matemático griego del siglo III 
llamado \textbf{Diófanto de Alejandría}.  
En su obra \textit{Arithmetica}, Diófanto propuso y resolvió problemas que buscaban soluciones enteras o fraccionarias positivas.  
Fue uno de los primeros en usar un lenguaje simbólico rudimentario, lo que le valió el título de \textit{“padre del álgebra”}.  
\end{reseñaplana}

\section*{Un enigma que viajó por los siglos.}
\begin{reseñaplana}
Durante la Edad Media y el Renacimiento, las ecuaciones diofánticas se convirtieron en un campo de desafíos intelectuales.  
Los matemáticos indios y árabes desarrollaron métodos ingeniosos para resolverlas, y más tarde los europeos retomaron la tradición.  
Entre ellos, el francés \textbf{Pierre de Fermat} dejó una nota al margen de su copia del libro de Diófanto con una afirmación que marcaría la historia:  
> “Es imposible dividir un cubo en dos cubos... y he descubierto una demostración maravillosa, pero este margen es demasiado estrecho para contenerla.”  
Así nació el famoso \textbf{Último Teorema de Fermat}, una ecuación diofántica que tardó más de 350 años en resolverse.  
En 1994, el matemático británico \textbf{Andrew Wiles} demostró el teorema utilizando herramientas modernas de teoría de números y 
geometría algebraica, cerrando uno de los capítulos más largos de la historia matemática.  
\end{reseñaplana} 

\section*{Mas allá de los enteros.}
\begin{reseñaplana}
Hoy, las ecuaciones diofánticas ya no son solo un pasatiempo de genios: son un pilar en la \textbf{teoría de números}, 
la rama de las matemáticas que estudia las propiedades de los enteros.  
Resolver una ecuación diofántica significa encontrar no cualquier número, sino uno que encaje perfectamente, sin decimales, sin aproximaciones.  
Su estudio ha llevado a descubrimientos profundos sobre la estructura de los números, la simetría y la naturaleza misma del infinito.  

Incluso las computadoras modernas, con toda su potencia, encuentran límites en este terreno: muchos problemas diofánticos son tan complejos 
que no existe un método general que garantice resolverlos todos. 
\end{reseñaplana} 

\section*{Del misterio antiguo a la seguridad digital.}
\begin{reseñaplana}
En el siglo XXI, las ecuaciones diofánticas tienen aplicaciones inesperadas y poderosas.  
Son la base de la \textbf{criptografía moderna}, el lenguaje que protege contraseñas, transacciones bancarias y 
comunicaciones en internet.  
Los sistemas de cifrado usan propiedades de números enteros enormes —como los que surgen de ecuaciones diofánticas— para crear 
códigos prácticamente imposibles de romper.  
Además, aparecen en la \textbf{inteligencia artificial}, en la búsqueda de patrones numéricos, 
y en la \textbf{teoría de la complejidad computacional}, donde ayudan a entender qué problemas puede o no puede resolver una máquina.  
De los papiros antiguos al ciberespacio, las ecuaciones diofánticas son un recordatorio de que las preguntas más 
simples —¿qué números encajan perfectamente?— pueden esconder los secretos más profundos del universo matemático.
\end{reseñaplana}

\section*{¿Qué vas a aprender?}
\begin{aprende}
  \item Comprender 
\end{aprende}

\vspace{1cm}
\begin{flushright}
  {\oneptup\itshape ``Es más fácil dar problemas que resolverlos.''}\\
  {\oneptup — Diofanto de Alejandría}
\end{flushright}


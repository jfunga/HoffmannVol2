% !TEX root = ../../main.tex
% =============================
% Intro del capítulo
% =============================

% --- Subtítulo ---
\vspace{0.5cm}
\subtitulocapitulo{Del trazo al territorio: soluciones bajo la parábola.}

\vspace{1em}
% --- Reseñas históricas ---
\section*{El equilibrio entre curvas y desigualdades.}

\begin{reseñaplana}
Las inecuaciones existen desde los orígenes de la matemática, pero durante siglos se trabajaron solo con líneas rectas.  
Los \textbf{babilonios} y \textbf{griegos} resolvían problemas de comparación de magnitudes y áreas, 
sin imaginar que algún día podrían hacerlo con curvas.  
El salto llegó con el estudio de las \textbf{ecuaciones cuadráticas}, aquellas donde aparece el término $x^2$.  

Cuando los matemáticos comenzaron a entender las parábolas —esas curvas que abren hacia arriba o hacia abajo—, 
descubrieron que también podían comparar regiones del plano: zonas donde la parábola estaba “por encima” o “por debajo” del eje.  
Ahí nació la idea de una \textbf{inecuación cuadrática}: una comparación no entre números, sino entre formas.  
\end{reseñaplana}

\section*{De los cuadrados al razonamiento visual.}
\begin{reseñaplana}
Durante el Renacimiento, el estudio de las ecuaciones cuadráticas permitió avanzar en la comprensión de sus desigualdades.  
Los matemáticos empezaron a analizar los signos de las expresiones cuadráticas, identificando intervalos donde una parábola 
tomaba valores positivos o negativos.  

Con la llegada del álgebra simbólica y de la geometría analítica en el siglo XVII, gracias a \textbf{Descartes} y \textbf{Fermat}, 
las inecuaciones cuadráticas se volvieron un tema visual: resolverlas era, literalmente, encontrar dónde una curva 
estaba sobre o bajo el eje $x$.  
El razonamiento geométrico y algebraico se unieron, y la resolución de inecuaciones dejó de ser una tarea puramente numérica.  
\end{reseñaplana}

\section*{El análisis del signo y las regiones del plano.}
\begin{reseñaplana}
Hoy, las inecuaciones cuadráticas son una herramienta fundamental para comprender el comportamiento de las funciones.  
Resolver una inecuación del tipo $ax^2 + bx + c > 0$ significa determinar qué valores de $x$ hacen que la parábola esté 
por encima del eje —una idea que une visualización y lógica matemática.  

En la enseñanza moderna, el análisis del signo del trinomio cuadrado se usa no solo para resolver desigualdades, 
sino también para entender intervalos de crecimiento, concavidad y extremos de las funciones.  
El estudiante aprende a “leer” una parábola: sus raíces, su vértice y las regiones donde domina el signo positivo o negativo.  
\end{reseñaplana}

\section*{De la física al modelado digital} 
\begin{reseñaplana} 
Las inecuaciones cuadráticas están presentes en todos los ámbitos donde se analizan límites o condiciones de seguridad.  
En \textbf{física}, sirven para determinar rangos de movimiento, energía o estabilidad en sistemas con trayectorias parabólicas.  
En \textbf{ingeniería}, ayudan a definir zonas seguras de carga o resistencia de materiales.  
En \textbf{economía}, modelan beneficios y pérdidas dentro de márgenes específicos de producción.  
En la era digital, se aplican en la \textbf{programación de videojuegos} y en \textbf{simulaciones físicas}, 
para determinar colisiones, trayectorias y áreas de acción.  
Incluso en \textbf{inteligencia artificial} y \textbf{optimización}, las inecuaciones cuadráticas definen regiones 
factibles donde se buscan soluciones eficientes.  

De las tablillas babilonias al código moderno, las inecuaciones cuadráticas siguen enseñándonos a ver el equilibrio 
entre lo posible y lo imposible, entre lo que está “por encima” y lo que está “por debajo” en el lenguaje eterno de las matemáticas.
\end{reseñaplana}

\section*{¿Qué vas a aprender?}
\begin{aprende}
  \item Comprender 
\end{aprende}

\vspace{1cm}
\begin{flushright}
  {\oneptup\itshape ``No debe prestarse atención a que el álgebra y la geometría parezcan distintas:\\ los hechos algebraicos son hechos geométricos demostrados.''}\\
  {\oneptup — Omar Khayyam}
\end{flushright}
% !TEX root = ../main.tex

% ==============================
% Temas del Libro
% ==============================

\clearpage
\thispagestyle{empty}

{ % --- bloque aislado ---

\begin{center}
  \vspace*{0.8cm}
  {\bfseries\MakeUppercase{\fontsize{16}{18}\selectfont TEMAS DEL LIBRO}} \\[0.5ex]
  \rule{\textwidth}{0.6pt}
  \vspace{0.6cm}
\end{center}

{\large   % <--- tamaño un punto más grande
\noindent
% === Tabla de temas (columna derecha desplazada 1 cm a la izquierda) ===
% === Tabla de temas (columna derecha desplazada solo 1 cm real) ===
% === Tabla de temas (columna derecha movida 5 cm a la derecha) ===
% === Tabla de temas (toda la tabla desplazada 0.5 cm a la derecha) ===
\hspace*{0.5cm}
\begin{tabular}{@{}llr@{}}
\tocmath{\pi} & Números irracionales              & \hspace*{4cm}11 \\[1.5ex]
\tocmath{a^n} & Potenciación                      & \hspace*{4cm}15 \\[1.5ex]
\tocmath{\sqrt[\leftroot{-2}\uproot{4}n]{\ }} & Radicación & \hspace*{4cm}33 \\[1.5ex]
\tocmath{\tfrac{1}{\sqrt{\ }}} & Racionalización           & \hspace*{4cm}51 \\[1.5ex]
\tocmath{\sqrt{x}}     & Ecuaciones irracionales           & \hspace*{4cm}91 \\[1.5ex]
\tocmath{x^2}          & Ecuación de segundo grado         & \hspace*{4cm}105 \\[1.5ex]
\tocmath{\{\}}         & Sistemas de ecuaciones            & \hspace*{4cm}145 \\[1.5ex]
\tocmath{\geq}         & Inecuaciones                      & \hspace*{4cm}185 \\[1.5ex]
\tocmath{|x|}          & Valor absoluto                    & \hspace*{4cm}215 \\[1.5ex]
\tocmath{\mathbb{R}^2} & El plano real                     & \hspace*{4cm}235 \\[1.5ex]
\tocmath{f(x)}         & Funciones                         & \hspace*{4cm}255 \\[1.5ex]
\tocmath{\leq}         & Inecuaciones cuadráticas          & \hspace*{4cm}295 \\[1.5ex]
\tocmath{\mathbb{D}}   & Ecuaciones diofánticas            & \hspace*{4cm}325 \\[1.5ex]
\tocmath{\triangle}    & Geometría (Pitágoras, Tales, Euclides) & \hspace*{4cm}355 \\[1.5ex]
\end{tabular}
}

\vspace{0.6cm}
\begin{center}
  {\large\textit{(El índice completo y detallado se encuentra al final del libro)}}
\end{center}

} % --- fin del bloque aislado ---

\clearpage

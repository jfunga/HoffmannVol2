% !TEX root = ../../main.tex
\begin{capitulobox}
El crecimiento lineal es aburrido… lo nuestro es \textbf{\textit{exponencial.}}
\end{capitulobox}

% =============================
% Capítulo 2 — Potenciación
% =============================

\subtitulocapitulo{Potencia}

Llamaremos potencia \emph{n}-sima de una cantidad al resultado de
multiplicar \emph{n} veces esa cantidad por sí misma.

Por ejemplo, la quinta potencia de 2, que indicaremos así:

\[
  2^5 \;=\;
  \llaveabajo{2 \cdot 2 \cdot 2 \cdot 2 \cdot 2}{5 veces}
  \;=\; 32
\]

Otros ejemplos:  

\begin{ejemplos2shortbull}(2)
  \task $x^4 = x \cdot x \cdot x \cdot x$
  \task $(-5)^3 = (-5)(-5)(-5) = -125$
\end{ejemplos2shortbull}

En una potencia se distinguen dos elementos: la \textit{base}, que es el
número que se multiplica por sí mismo, y el \textit{exponente}, que indica
cuántas veces esto se hace.

En la expresión \(2^n\), la base es 2 y el exponente es \(n\).

Dado que el producto de números positivos da como resultado un número
positivo, toda potencia con base positiva dará, al ser resuelta, una
cantidad positiva.  

En cambio, si la base es un número negativo, hay que distinguir dos
casos:

a) Si el exponente es \textit{impar}, la potencia será negativa: 

\begin{ejemplos2shortbull}(2)
  \task $(-7)^3 = (-7)(-7)(-7) = -343$
  \task $(-3)^5 = (-3)(-3)(-3)(-3)(-3) = -243$
\end{ejemplos2shortbull}


b) Si el exponente es \textit{par}, la potencia será positiva: 

\begin{ejemplos2shortbull}(2)
  \task $(-2)^4 = (-2)(-2)(-2)(-2) = 16$
  \task $(-11)^2 = (-11)(-11) = 121$
\end{ejemplos2shortbull}

Por convención, cuando el exponente es la unidad, ésta no se indica:

\begin{ejemplos1bullet}(1)
  \task $8^1 = 8$
  \contline{$(-4)^1 = -4$}
  \contline{$x^1 = x$}
\end{ejemplos1bullet}

% !TEX root = ../../../main.tex
% =============================
% EJERCICIO (1)
% =============================
\begin{BloqueEjercicios}[Potencias con exponentes enteros]
  \begin{ej3col}
    \item $2^6$
    \item $(-3)^2$
    \item $(-5)^4$
    \item $(-2)^3$
    \item $7^2$
    \item $(-6)^3$
    \item $0^7$
    \item $(-2)^5$
    \item $(-3)^3$
    \item $(-1)^{33}$
    \item $10^3$
    \item $(-1)^{14}$
  \end{ej3col}
\end{BloqueEjercicios}


Es importante tener en cuenta que, si una potencia va precedida por
algún signo, el resultado final queda afectado por dicho signo.  

Ejemplos:

\begin{ejemplos1bullet}(1)
  \task $(-2)^4 = (-2)(-2)(-2)(-2) = 16$
  \task $-(-2)^4 = -[(-2)(-2)(-2)(-2)] = -16$
  \task $-2^4 = -[2 \cdot 2 \cdot 2 \cdot 2] = -16$
  \task $(-3)^3 = (-3)(-3)(-3) = -27$
  \task $-(-3)^3 = -[(-3)(-3)(-3)] = 27$
  \task $-3^3 = -[3 \cdot 3 \cdot 3] = -27$
  \task $-a^3 = -[a \cdot a \cdot a] = -a^3$
  \task $-(-a)^3 = -[(-a)(-a)(-a)] = a^3$
\end{ejemplos1bullet}


% Ejercicio 2
% !TEX root = ../../../main.tex
% =============================
% EJERCICIO (2)
% =============================

\begin{BloqueEjercicios}[Signos y potencias con paréntesis]
  \begin{ej3col}
    \item $-(-1)^4$
    \item $-(-1)^5$
    \item $-(-1)^6$
    \item $-(-4)^3$
    \item $-4^2$
    \item $(-4)^2$
    \item $-(-x)^4$
    \item $-8^2$
    \item $(-8)^2$
    \item $-(-8)^2$
  \end{ej3col}
\end{BloqueEjercicios}


% Caso 1 (enseñanzas del profesor)
% !TEX root = ../../../main.tex

% =============================
% CASO 1
% =============================

%\begin{Caso}[Cálculo de expresión con potencias de distintos signos]
  %\casolinea{Calcular la siguiente expresión:}{(-4)^2 - (-2)^3 + (-3)^3 - (-1)^{10}}
  %\casolinea{Recuerda que las potencias pares dan positivo, mientras que las impares conservan el signo.}{= [16] - [-8] + [-27] - [1]}
  %\casolinea{Eliminamos los signos de agrupación:}{= 16 + 8 - 27 - 1}
  %\casolinea{Sumando los términos obtenemos:}{\Resultado{-4}}
%\end{Caso}


% =============================
% CASO 1 - PRUEBA DE A POCO
% =============================

\begin{Caso}[Cálculo de expresión con potencias de distintos signos]
  \casolinea{Calcular la siguiente expresión:}{(-4)^2 - (-2)^3 + (-3)^3 - (-1)^{10}}

  % NUEVA micro-observación (una línea)
  \casolinea{Recuerda: potencias \textit{pares} dan resultado positivo; potencias \textit{impares} conservan el signo.}{}

  \casolinea{Calculamos primero cada potencia (encerramos en corchetes los resultados):}{= [16] - [-8] + [-27] - [1]}
  \casolinea{Eliminamos los signos de agrupación:}{= 16 + 8 - 27 - 1}

  % CAMBIO: usar \Resultado{-4} en vez de \fbox{$-4$}
  \casolinea{Sumamos los términos:}{= \Resultado{-4}}
\end{Caso}












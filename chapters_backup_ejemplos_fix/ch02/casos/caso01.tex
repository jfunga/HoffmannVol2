% !TEX root = ../../../main.tex

% =============================
% CASO 1
% =============================

%\begin{Caso}[Cálculo de expresión con potencias de distintos signos]
  %\casolinea{Calcular la siguiente expresión:}{(-4)^2 - (-2)^3 + (-3)^3 - (-1)^{10}}
  %\casolinea{Recuerda que las potencias pares dan positivo, mientras que las impares conservan el signo.}{= [16] - [-8] + [-27] - [1]}
  %\casolinea{Eliminamos los signos de agrupación:}{= 16 + 8 - 27 - 1}
  %\casolinea{Sumando los términos obtenemos:}{\Resultado{-4}}
%\end{Caso}



% =============================
% CASO 1 - PRUEBA DE A POCO
% =============================

\begin{Caso}[Cálculo de expresión con potencias de distintos signos]
  \casolinea{Calcular la siguiente expresión:}{(-4)^2 - (-2)^3 + (-3)^3 - (-1)^{10}}
  \casolinea{Calculamos, en primer lugar, cada una de las \\ potencias
  señaladas encerrando en corchetes \\ los resultados:}{= [16] - [-8] + [-27] - [1]}
  \casolinea{Eliminamos los signos de agrupación:}{= 16 + 8 - 27 - 1}
  \casolinea{Sumando los términos obtenemos:}{= \fbox{$-4$}}
\end{Caso}






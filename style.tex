% !TEX root = main.tex
% =============================
% style.tex — Estilo del libro
% Proyecto: Math Hoffmann — Vol. 2
% Compilación: XeLaTeX (latexmk -xelatex)
% =============================

% -------------------------------------------------
% 0) MOTOR, IDIOMA, TIPOGRAFÍA Y MATEMÁTICAS (ROBUSTO)
% -------------------------------------------------

% ======= Build switches (modo borrador/producción, externalización TikZ) =======
\newif\ifdraftbuild      % \draftbuildtrue   -> borrador (más rápido)
\draftbuildfalse         % \draftbuildfalse  -> producción
\newif\ifexttikz         % externalizar TikZ (requiere -shell-escape)
\exttikzfalse

% ======= Motor tipográfico y matemático =======
\usepackage{fontspec}          % XeLaTeX/LuaLaTeX
\usepackage{mathtools}         % (carga amsmath)
\usepackage{unicode-math}      % Matemáticas Unicode

% ======= Idioma: SOLO ESPAÑOL =======
\usepackage{polyglossia}
\setdefaultlanguage{spanish}

% Helpers de contenido (solo español activo)
\newcommand{\ES}[1]{#1}
\newcommand{\EN}[1]{}

% Etiquetas (solo español)
\newcommand{\ChapterName}{Capítulo}

% ======= Fuentes con fallback seguro =======
\setmainfont{TeX Gyre Termes}  % Texto
\setmathfont{STIX Two Math}    % Matemáticas
\IfFontExistsTF{TeX Gyre Heros}{\setsansfont{TeX Gyre Heros}}{}
\IfFontExistsTF{TeX Gyre Cursor}{\setmonofont{TeX Gyre Cursor}}{}

% ======= Microtipografía y espaciado editorial =======
\usepackage{microtype}
\SetTracking{encoding=*}{40}       % tracking suave
\frenchspacing                     % espaciado tras punto consistente

% Párrafo: sin sangría + espacio entre párrafos
\setlength{\parindent}{0pt}
\setlength{\parskip}{0.8em}

% Control viudas/huérfanas (suaviza composiciones largas)
\clubpenalty=10000   % evita viudas
\widowpenalty=10000  % evita huérfanas
\displaywidowpenalty=10000


% -------------------------------------------------
% 1) PÁGINA, COLORES, GRÁFICOS, ENLACES
% -------------------------------------------------

% Geometría de página (ajusta márgenes a tu estándar)
\usepackage[letterpaper,margin=1in]{geometry}

% Colores amplios (tablas incluidas)
\usepackage[dvipsnames,svgnames,table]{xcolor}

% Gráficos y rutas
\usepackage{graphicx}
\graphicspath{{media/}{media/ch01/}{media/ch02/}} % añade más según crezcan

% Gráficos y rutas
\usepackage{graphicx}
\graphicspath{{media/}{media/ch01/}{media/ch02/}} % añade más según crezcan

% --- Dibujo y TikZ (necesario para llaves/decorations) ---
\usepackage{tikz}
\usetikzlibrary{calc,arrows.meta,positioning,decorations.pathreplacing}
% Si en algún punto usas marcas/flows, puedes añadir:
% \usetikzlibrary{decorations.markings}

% Modo borrador para imágenes (rápido en builds largos)
\ifdraftbuild
  \PassOptionsToPackage{draft}{graphicx}
\fi

% Hipervínculos y metadatos PDF (cargar al final del preámbulo)
\usepackage[hidelinks]{hyperref}
\hypersetup{
  pdftitle   = {Math Hoffmann — Volumen},
  pdfauthor  = {Jorge Gid Hoffmann / J&J Socrates, LLC},
  pdfsubject = {Matemáticas — Educación Secundaria},
  pdfkeywords= {Matemáticas, Bilingüe, Educación},
  pdfcreator = {XeLaTeX},
  unicode    = true
}

% Modo borrador para imágenes (rápido en builds largos)
\ifdraftbuild
  \PassOptionsToPackage{draft}{graphicx}
\fi

% Hipervínculos y metadatos PDF (cargar al final del preámbulo)
\usepackage[hidelinks]{hyperref}
\hypersetup{
  pdftitle   = {Math Hoffmann — Volumen},
  pdfauthor  = {Jorge Gid Hoffmann / J&J Socrates, LLC},
  pdfsubject = {Matemáticas — Educación Secundaria},
  pdfkeywords= {Matemáticas, Bilingüe, Educación},
  pdfcreator = {XeLaTeX},
  unicode    = true
}

% Si usas cleveref más adelante, cárgalo aquí (opcional):
% \usepackage[nameinlink]{cleveref}


% -------------------------------------------------
% 2) NÚMEROS, LISTAS Y UTILIDADES (sin 'locale')
% -------------------------------------------------

\usepackage{siunitx}

\sisetup{
  output-decimal-marker = {,}, % coma decimal
  group-separator       = {.}, % punto de millares
  per-mode              = symbol,
  detect-all
}

\usepackage{enumitem}
\setlist{topsep=.5ex,itemsep=.25ex,parsep=0pt,partopsep=0pt}

% (Opcional) Citas “inteligentes”
% \usepackage{csquotes} % si lo activas, usa \enquote{...}


% -------------------------------------------------
% 3) ENCABEZADOS Y PIES (fancyhdr)
% (usa \href→ hyperref ya cargado)
% -------------------------------------------------
\usepackage{fancyhdr}
\setlength{\headheight}{14pt} % evita warnings aunque no haya encabezado
\pagestyle{fancy}
\fancyhf{}
\renewcommand{\headrulewidth}{0pt}

% URL centralizada (mantener en un solo sitio)
\providecommand{\siteurl}{https://www.mathhoffmann.com}

% Macro para título de capítulo en pie (se actualiza en \fullchapter)
\newcommand{\chaptertitle}{} 

% Páginas pares → número + web
\fancyfoot[LE]{%
  \rule[0.5ex]{\linewidth}{0.4pt}\\
  \small \thepage \quad \textit{visita} → \href{\siteurl}{www.mathhoffmann.com}%
}

% Páginas impares → símbolo+capítulo + número
\fancyfoot[RO]{%
  \rule[0.5ex]{\linewidth}{0.4pt}\\
  \small \chaptertitle \enspace \thepage%
}

% Estilos vacíos para portada/créditos/etc.
\usepackage{etoolbox} % utilidades varias (ok tenerlo aquí)
\fancypagestyle{emptyplain}{%
  \fancyhf{}%
  \renewcommand{\headrulewidth}{0pt}%
  \renewcommand{\footrulewidth}{0pt}%
}
\fancypagestyle{plain}{%
  \fancyhf{}%
  \renewcommand{\headrulewidth}{0pt}%
  \fancyfoot[RO]{\rule[0.5ex]{\linewidth}{0.4pt}\\ \small \chaptertitle \enspace \thepage}%
}

% -------------------------------------------------
% 4) TITULACIÓN (chapters/sections) — Aparición temprana
% -------------------------------------------------
\usepackage{titlesec}

% Bloque de capítulo (símbolo + título) y marcas para pies
% Uso: \fullchapter{Nº}{Símbolo}{Título}
\newcommand{\fullchapter}[3]{%
  \chapter*{}%
  \addcontentsline{toc}{chapter}{#3}%
  \markboth{#2\; #3}{}%
  \renewcommand{\chaptertitle}{#2\; #3}%
  \begin{center}
    \vspace*{-2cm} % ajuste fino vertical
    {\fontsize{22}{24}\selectfont\bfseries #1}\par\vspace{1ex}
    {\fontsize{20}{22}\selectfont\bfseries \scalebox{1.4}{#2} \quad #3}\par\vspace{0.2ex}
    \rule{\linewidth}{1pt}\\[-1.2ex]
    \rule{0.85\linewidth}{0.5pt}
  \end{center}%
}

% Subtítulo de capítulo (15pt bold)
\newcommand{\subtitulocapitulo}[1]{%
  \vspace{1.5ex}%
  \noindent{\fontsize{15}{17}\selectfont\bfseries #1}\par\vspace{1ex}%
}

% Secciones / Subsecciones
\titleformat{\section}
  {\normalfont\bfseries\fontsize{14}{16}\selectfont}{\thesection}{1.5em}{}
\titlespacing*{\section}{0pt}{1.5ex}{1ex}

\titleformat{\subsection}
  {\normalfont\bfseries\fontsize{12}{14}\selectfont}{\thesubsection}{1em}{}
\titlespacing*{\subsection}{0pt}{1ex}{0.5ex}

% -------------------------------------------------
% 5) INTRO DE CAPÍTULO — listas y cita manual
% (la cita ya la haces en intro.tex con flushright; aquí solo helpers)
% -------------------------------------------------
\usepackage{enumitem,pifont}
\setlist[itemize]{itemsep=0.3ex, topsep=0.5ex}

% Lista “¿Qué vas a aprender?”
\newlist{aprende}{itemize}{1}
\setlist[aprende]{label=\ding{228}, leftmargin=2em, itemsep=.25em, topsep=.6em}

% +1pt para la cita (adaptativo al tamaño actual)
\makeatletter
\newcommand{\oneptup}{%
  \edef\mh@size{\strip@pt\dimexpr\f@size pt + 1pt\relax}%
  \edef\mh@base{\strip@pt\dimexpr\f@size pt + 3pt\relax}%
  \fontsize{\mh@size}{\mh@base}\selectfont
}
\makeatother

% Macros mínimos de portada (si los usas en apertura)
\newcommand{\MHSizeSub}[1]{{\itshape\fontsize{15}{18}\selectfont #1}}
\newcommand{\MHSizeAuthor}[1]{{\fontsize{15}{18}\selectfont #1}}
\newcommand{\AuthorSC}[1]{{\textsc{#1}}}

% -------------------------------------------------
% 6) MATEMÁTICAS, ENTORNOS Y ESPACIADOS
% -------------------------------------------------
\usepackage{empheq}
\usepackage{siunitx}     % números/decimales
\usepackage{tasks}       % listas tipo ejercicios en columnas
\usepackage{multicol}
\usepackage{relsize}
\usepackage{soul}        % subrayados
\usepackage{changepage}  % márgenes locales
\usepackage{array,tabularx}
\usepackage{amsmath} 

% Espaciado vertical en ecuaciones (compacto)
\makeatletter
\g@addto@macro\normalsize{%
  \setlength\abovedisplayskip{3pt}%
  \setlength\belowdisplayskip{3pt}%
  \setlength\abovedisplayshortskip{2pt}%
  \setlength\belowdisplayshortskip{2pt}%
}
\makeatother

% Notación rápida en TOC (símbolos grandes)
\newcommand{\tocmath}[1]{{\Large $#1$}}

% Números y decimales (ES/EN)
\sisetup{
  output-decimal-marker = {,}, % decimal = coma
  group-separator = {.},       % miles = punto
  group-minimum-digits = 4
}
\newcommand{\numes}[1]{%
  \begingroup
  \sisetup{output-decimal-marker = {,}, group-separator = {}, group-minimum-digits = 4}%
  \num{#1}%
  \endgroup
}
\newcommand{\numen}[1]{%
  \begingroup
  \sisetup{output-decimal-marker = {.}, group-separator = {,}}%
  \ensuremath{\num{#1}}%
  \endgroup
}

% Notación decimal periódica
\newcommand{\periodo}[1]{\overline{#1}}
\newcommand{\periodoen}[1]{\overline{#1}}
\newcommand{\periodico}[2]{%
  \numes{#1}\;#2\;#2\;#2\ldots \;=\; \numes{#1}\overline{#2}%
}

% -------------------------------------------------
% 7) CAJAS Y BLOQUES (tcolorbox) — usados en cuerpo
% -------------------------------------------------
\usepackage[most]{tcolorbox}
\tcbuselibrary{theorems,skins}

% Quote redefinido (estilo suave)
\renewenvironment{quote}{%
  \begin{tcolorbox}[
    enhanced,
    colback=black!5,
    colframe=white,
    boxrule=0pt,
    left=2cm, right=2cm,
    sharp corners,
    top=4pt, bottom=4pt,
    fontupper=\itshape
  ]%
}{%
  \end{tcolorbox}
}

% Bloques varios
\newenvironment{reseñaplana}{\par\noindent\ignorespaces}{\par}
\newenvironment{reseñaitem}{%
  \renewcommand\labelitemi{\ding{228}}%
  \begin{itemize}[leftmargin=2em]%
}{\end{itemize}}
\newenvironment{introblock}[1]{%
  \vspace{2ex}\noindent\textbf{\MakeUppercase{#1}}%
  \begin{quote}\itshape
}{\end{quote}\vspace{2ex}}
\newenvironment{temablock}{%
  \begin{tcolorbox}[colback=white,colframe=black!40,boxrule=0.4pt,
    sharp corners, enhanced, breakable, title={Explicación}]%
}{\end{tcolorbox}}
\newenvironment{cierreblock}{%
  \begin{tcolorbox}[colback=black!5,colframe=black!80,boxrule=0.8pt,
    sharp corners, enhanced, breakable, title={Reflexión final}]%
}{\end{tcolorbox}}

% Caja apertura de capítulo (si se usa)
% --- Definición de la caja ---
\newtcolorbox{capitulobox}{%
  enhanced, breakable,
  colback=yellow!10, colframe=black, boxrule=0.6pt,
  arc=6pt, width=\textwidth,
  fontupper=\itshape,
  before skip=12pt, after skip=24pt,
  before upper={\begingroup\begin{center}\begin{minipage}{0.9\textwidth}\centering\sloppy\capituloboxPlusTwo},
  after upper={\end{minipage}\end{center}\endgroup}
}

% --- Helper justo debajo (perfectamente válido) ---
\makeatletter
\newcommand{\capituloboxPlusTwo}{%
  \fontsize{\dimexpr\f@size pt + 2pt\relax}{\baselineskip}\selectfont}
\makeatother

% ==============

% Caja de definición
\newtcolorbox{defbox}{%
  enhanced, breakable,
  colback=white, colframe=black!60, boxrule=0.6pt, arc=6pt,
  left=8pt, right=8pt, top=8pt, bottom=8pt,
  before skip=10pt, after skip=10pt,
  width=0.9\textwidth, center
}

% Caja extra/aparte
\newtcolorbox{extrabox}{%
  enhanced, breakable,
  colback=white, colframe=black!70, boxrule=0.8pt, arc=4pt,
  width=0.95\textwidth, center,
  left=10pt, right=10pt, top=10pt, bottom=10pt,
  before skip=16pt, after skip=16pt
}

% -------------------------------------------------
% 8) EJERCICIOS — bloque + entorno unificado
% -------------------------------------------------

% Contador global (igual que ya tenías)
\newcounter{ejercicioGlobal}
\renewcommand{\theejercicioGlobal}{\arabic{ejercicioGlobal}}

% Bloque decorativo para enmarcar un conjunto de ejercicios
\NewDocumentEnvironment{BloqueEjercicios}{ O{} }{%
  \refstepcounter{ejercicioGlobal}%
  \begin{tcolorbox}[
    enhanced,
    colback=white, colframe=black!60, boxrule=0.6pt,
    rounded corners, arc=2pt,
    left=1em, right=1em, top=1em, bottom=1em,
    title={Ejercicio~\theejercicioGlobal \IfNoValueTF{#1}{}{:\ #1}},
    fonttitle=\bfseries, coltitle=black,
    attach boxed title to top left={yshift=-2mm, xshift=2mm},
    boxed title style={colback=black!10, colframe=black!60, sharp corners,
      boxrule=0.4pt, top=0.6mm, bottom=0.6mm, left=1mm, right=1mm}
  ]%
}{%
  \end{tcolorbox}%
}

% Helpers compartidos (seguros y locales)
\providecommand{\task}{\item}             % por si alguien usa \task fuera
\providecommand{\contline}[1]{\\#1}       % línea de continuación
\providecommand{\MathItems}{}             % opcional: convierte \item en modo display si lo activas
\providecommand{\MHcolsep}{1.8em}         % separación estándar entre columnas

% ===== Entorno UNIFICADO de EJERCICIOS =====
% Uso: \begin{ejercicios}[<cols>][<opciones enumitem>]
%   <cols> por defecto = 1; si pones 2,3,4 activa multicol
%   <opciones enumitem> (opcional): por si quieres ajustar labelsep, noitemsep, etc.
\NewDocumentEnvironment{ejercicios}{ O{1} O{} }{%
  \setlength{\columnsep}{\MHcolsep}%
  \ifnum#1>1\begin{multicols}{#1}\fi
  \begin{enumerate}[label=\arabic*), leftmargin=1.8em, itemsep=1.2ex, topsep=1ex, parsep=1ex, #2]%
    \let\task\item
    \MathItems
}{%
  \end{enumerate}%
  \ifnum#1>1\end{multicols}\fi
}

% ---- Wrappers de compatibilidad (no toques capítulos) ----
% Definimos (no "renovamos") los atajos hacia el entorno principal 'ejercicios'
\NewDocumentEnvironment{ej1col}{}{ \begin{ejercicios}[1] }{ \end{ejercicios} }
\NewDocumentEnvironment{ej2col}{}{ \begin{ejercicios}[2] }{ \end{ejercicios} }
\NewDocumentEnvironment{ej3col}{}{ \begin{ejercicios}[3] }{ \end{ejercicios} }
\NewDocumentEnvironment{ej4col}{}{ \begin{ejercicios}[4] }{ \end{ejercicios} }

% Alias "automático": 2 columnas por defecto (para compatibilidad con tu código)
\NewDocumentEnvironment{ejAuto}{}{ \begin{ejercicios}[2] }{ \end{ejercicios} }
\NewDocumentEnvironment{ejcol}{}{  \begin{ejercicios}[2] }{ \end{ejercicios} }

% -------------------------------------------------
% 9) EJEMPLOS — entorno unificado (multicol + enumitem)
% -------------------------------------------------

% ===== Entorno UNIFICADO de EJEMPLOS =====
% Uso: \begin{ejemplos}[<cols>][<label>][<opciones enumitem>]
%   <cols>   : 1 por defecto (si >1 activa multicol)
%   <label>  : por defecto \textbullet  (pon (\alph*) si quieres a), b), c)…)
%   <opciones enumitem> : ajustes finos (opcional)
\NewDocumentEnvironment{ejemplos}{ O{1} O{\textbullet} O{} }{%
  \setlength{\columnsep}{\MHcolsep}%
  \ifnum#1>1\begin{multicols}{#1}\fi
  \begin{enumerate}[label=#2, labelsep=0.6em,
                    leftmargin=3.5em, itemsep=1.6ex, topsep=2ex, #3]%
    \let\task\item
}{%
  \end{enumerate}%
  \ifnum#1>1\end{multicols}\fi
}

% ---- Wrappers de compatibilidad (no toques capítulos) ----
% 1 columna
\NewDocumentEnvironment{ejemplos1}{}{ \begin{ejemplos}[1][(\alph*)] }{ \end{ejemplos} }
\NewDocumentEnvironment{ejemplos1bullet}{}{ \begin{ejemplos}[1][\textbullet] }{ \end{ejemplos} }

% 2 columnas
\NewDocumentEnvironment{ejemplos2short}{}{ \begin{ejemplos}[2][(\alph*)] }{ \end{ejemplos} }
\NewDocumentEnvironment{ejemplos2shortbull}{}{ \begin{ejemplos}[2][\textbullet] }{ \end{ejemplos} }
\NewDocumentEnvironment{ejemplos2large}{}{ \begin{ejemplos}[2][(\alph*)] }{ \end{ejemplos} }

% 3 columnas
\NewDocumentEnvironment{ejemplos3}{}{ \begin{ejemplos}[3][(\alph*)] }{ \end{ejemplos} }
\NewDocumentEnvironment{ejemplos3bullet}{}{ \begin{ejemplos}[3][\textbullet] }{ \end{ejemplos} }


% -------------------------------------------------
% 10) CASOS Y TABLAS AUXILIARES
% -------------------------------------------------

% --- Contador de casos: reinicia en cada capítulo ---
\newcounter{caso}[chapter]
\renewcommand{\thecaso}{\arabic{caso}} % Muestra 1,2,3... (sin prefijo de capítulo)

% ===== Caso (variante ancho completo) =====
\NewDocumentEnvironment{CasoL}{ O{} }{%
  \refstepcounter{caso}%
  \par\medskip
  \noindent
  \tikz[baseline=(X.base)]{
    \node[
      fill=yellow!35, rounded corners=1pt,
      inner xsep=4pt, inner ysep=3pt,
      text height=2ex, text depth=.5ex, anchor=base west
    ] (X) {\textbf{\Large Caso \thecaso}%
      \if\relax\detokenize{#1}\relax\else~–~\textbf{#1}\fi};
  }%
  \par\vspace{1em}
}{%
  \par\vspace{1em}
}

% Alias para compatibilidad: 'Caso' se comporta como 'CasoL'
\NewDocumentEnvironment{Caso}{O{}}
  {\begin{CasoL}[#1]}
  {\end{CasoL}}

% Línea de Caso con contenido ANCHO COMPLETO debajo del texto
% Uso normal: \casolineaW{texto}{expresión/figura}
% Uso con * : \casolineaW*{texto}{bloque ya en display (aligned, tikzpicture, etc.)}
\NewDocumentCommand{\casolineaW}{s m m}{%
  \noindent\hspace*{1.5em}%
  \begin{minipage}{0.92\linewidth}
    #2\par\vspace{0.4ex}%
    \noindent
    \IfBooleanTF{#1}{%
      % versión con *: NO envolver en \[ \]
      #3%
    }{%
      % versión sin *: envolver en display math
      \[
        \displaystyle #3
      \]%
    }%
  \end{minipage}\par
}

% Variante dos columnas: texto a la izquierda, matemática a la derecha
% Uso normal: \casolinea{texto}{expresión}
% Uso con *   : \casolinea*{texto}{bloque en display (aligned, tikzpicture, etc.)}
\NewDocumentCommand{\casolinea}{s m m}{%
  \noindent
  \begin{minipage}[t]{0.58\linewidth}
    #2%
  \end{minipage}\hfill
  \begin{minipage}[t]{0.38\linewidth}
    \IfBooleanTF{#1}{%
      #3%
    }{%
      \[
        \displaystyle #3
      \]%
    }%
  \end{minipage}\par\vspace{0.6ex}%
}


% -------------------------------------------------
% 11) UTILIDADES GENERALES
% -------------------------------------------------
% Esquemas/figuras centradas con título opcional
\newcommand{\EsquemaConjuntos}[2][]{%
  \vspace{0.5cm}
  \begin{center}
    \includegraphics[width=0.9\linewidth]{#2}%
    \if\relax\detokenize{#1}\relax\else\\[4pt]\textbf{#1}\fi
  \end{center}
  \vspace{0.5cm}
}

% Fracciones en línea (ligeramente elevadas)
\newcommand{\nfrac}[3][2pt]{%
  \raisebox{#1}{$\dfrac{#2}{#3}$}\vphantom{p_{qgj}}%
}

% Llave decorativa inferior con texto
\newcommand{\llaveabajo}[2]{%
  \mathrel{\tikz[baseline=(m.base)]{
    \node[inner sep=0pt] (m) {$#1$};
    \draw [decorate,decoration={brace,mirror,amplitude=6pt}]
      (m.south west) -- (m.south east)
      node[midway,yshift=-9pt]{\scriptsize #2};
  }}%
}

% Estilo global de tablas
\renewcommand{\arraystretch}{1.2}
\setlength{\tabcolsep}{0.8em}

% Casos particulares (ej.: años, cap. 1)
\newcommand{\anios}{\text{años}}

% ===== Índice: macro de línea con icono a la derecha =====
\newcommand{\IndiceCap}[2]{%
  \noindent\textbf{#1}\hfill{\large\ensuremath{#2}}\\
}

% ===== Set de íconos (ajustables) para cada capítulo =====
% Nota: usa símbolos compactos (no expresiones largas) para evitar “listados raros”.
\newcommand{\iconIrracionales}{\pi}               % Cap. 1
\newcommand{\iconPotenciacion}{x^n}               % Cap. 2
\newcommand{\iconRadicacion}{\sqrt{\phantom{x}}}  % Cap. 3 (raíz “vacía” estética)
\newcommand{\iconRacionalizacion}{\dfrac{1}{\sqrt{\phantom{x}}}} % Cap. 4
\newcommand{\iconCuadratica}{x^2}                 % Cap. 5 (ícono simple para cuadrática)
\newcommand{\iconSistemas}{\{\}}                  % Cap. 6 (llave como “sistema” sin listar ecuaciones)
\newcommand{\iconInecuaciones}{x>a}               % Cap. 7
\newcommand{\iconValorAbsoluto}{|x|}              % Cap. 8
\newcommand{\iconPlano}{(x,y)}                    % Cap. 9
\newcommand{\iconFunciones}{f(x)}                 % Cap.10
\newcommand{\iconIneqCuad}{x^2>0}                 % Cap.11
\newcommand{\iconDiofanticas}{ax+by=c}            % Cap.12
\newcommand{\iconGeometria}{\triangle}            % Cap.13
\newcommand{\iconRecap}{\sum}                     % Cap.14


% --- Logo de portada: macro esperada por apertura/portada.tex ---
% Ruta del logo (respeta mayúsculas/minúsculas del archivo)
\providecommand{\MHLogoPath}{media/Apertura/Logo.png}

% \MHLogo[<ancho>]  -> por defecto 0.18\linewidth
\newcommand{\MHLogo}[1][0.18\linewidth]{%
  \IfFileExists{\MHLogoPath}{%
    \includegraphics[width=#1]{\MHLogoPath}%
  }{%
    % Si no se encuentra el archivo, no romper la compilación:
    \typeout{MHLogo: archivo no encontrado en \MHLogoPath}%
    \begingroup
      \setlength{\fboxsep}{1pt}%
      \fbox{\rule{#1}{0pt}\rule{0pt}{#1}}%
    \endgroup
  }%
}

% =============================
% FIN DE style.tex
% =============================


% --- Constantes matemáticas (Step 3) ---
\newcommand{\ee}{\mathrm{e}} % Constante de Euler en recta

% --- Paired delimiters (para valores absolutos, paréntesis grandes automáticos) ---
\DeclarePairedDelimiter\abs{\lvert}{\rvert}
\DeclarePairedDelimiter\paren{\lparen}{\rparen}

% --- Resultado enmarcado (uso: \Resultado{-4}) ---
\newcommand{\Resultado}[1]{\fbox{\(#1\)}}

% --- Nota del profesor (caja transversal) ---
\newtcolorbox{ProfeTip}[1][]{enhanced, breakable,
  colback=blue!3, colframe=blue!55!black, boxrule=.5pt, arc=2pt,
  left=1em, right=1em, top=.6em, bottom=.6em,
  borderline west={2pt}{0pt}{blue!60},
  fonttitle=\bfseries, title={Nota del profesor},
  #1
}

% ===== Listado exclusivo para "Importancia actual" =====
% Uso: \begin{importancialist} \item ... \end{importancialist}
\newlist{importancialist}{itemize}{1}
\setlist[importancialist]{%
  label=\ding{228},% marcador distinto a {aprende}
  leftmargin=*,    % alinear con el texto
  itemsep=.25em,   % separación entre ítems
  topsep=.2em,     % separación con párrafo previo
  parsep=0pt,
  partopsep=0pt,
  align=left
}
% =======================================================

% ---- Ajuste visual para la lista "Importancia actual" (cap. 5) ----
% Distinto al estilo de {aprende}: usa check ✔ y más aire entre ítems.
\setlist[importancialist]{%
  label=\ding{51},     % ✔ (pifont)
  labelsep=.55em,      % separación etiqueta-texto
  leftmargin=*,        % alinear al margen
  itemsep=.35em,       % aire entre ítems
  topsep=.25em,        % aire con el párrafo previo
  parsep=0pt,partopsep=0pt,
  align=left
}
% -------------------------------------------------------------------

% ===== Estilo global para listas genéricas (itemize) =====
% No afecta a listas personalizadas creadas con enumitem (aprende, reseñaitem, importancialist).
\setlist[itemize]{%
  label=\textbullet,  % bullet •
  leftmargin=*,       % alineado limpio al texto
  itemsep=.25em,      % aire entre ítems
  topsep=.25em,       % aire con el párrafo previo
  parsep=0pt, partopsep=0pt
}
% =========================================================

% =========================================================
% POLÍTICA DE PUNTUACIÓN — Math Hoffmann
% ---------------------------------------------------------
% No se coloca punto (.) ni ningún signo de puntuación al
% final de expresiones matemáticas, ni en modo inline ($...$)
% ni en display (\[...\]).
% Esto sigue el estilo editorial de Jorge Gid Hoffmann.
% =========================================================

% =========================================================
% POLÍTICA DE PUNTUACIÓN — Math Hoffmann
% ---------------------------------------------------------
% No se coloca punto (.) ni ningún signo de puntuación al
% final de expresiones matemáticas, ni en modo inline ($...$)
% ni en display (\[...\]).
% Esto sigue el estilo editorial de Jorge Gid Hoffmann.
% =========================================================



% Pegamento: elimina el salto de párrafo que añade \parskip
\newcommand{\IntroGlue}{\vspace{-\parskip}}










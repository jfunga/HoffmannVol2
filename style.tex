% !TEX root = main.tex
% =============================
% style.tex — Estilo del libro
% Proyecto: Math Hoffmann — Vol. 2
% Compilación: XeLaTeX (latexmk -xelatex)
% =============================

% -------------------------------------------------
% 0) MOTOR, IDIOMA, TIPOGRAFÍA Y MATEMÁTICAS
% -------------------------------------------------
\usepackage{fontspec}          % Fuentes (XeLaTeX/LuaLaTeX)
\usepackage{unicode-math}      % Matemáticas con Unicode

% Idioma (después de fontspec)
\usepackage{polyglossia}
\setmainlanguage{spanish}

% Fuentes del libro
\setmainfont{TeX Gyre Termes}  % Texto
\setmathfont{STIX Two Math}    % Matemáticas

% Párrafo: estilo editorial (espaciado entre párrafos, sin sangría)
\setlength{\parindent}{0pt}
\setlength{\parskip}{0.8em}

% -------------------------------------------------
% 1) DIMENSIONES DE PÁGINA Y ENLACES
% (geom. antes; hyperref near-last pero antes de usar \href)
% -------------------------------------------------
\usepackage[
  paperwidth=8.5in,
  paperheight=11in,
  inner=1.0in,
  outer=0.5in,
  top=0.75in,
  bottom=0.85in,
  headsep=0.25in,
  footskip=0.4in
]{geometry}

\usepackage[hidelinks]{hyperref} % enlaces limpios

% -------------------------------------------------
% 2) GRÁFICOS, COLOR Y DIBUJO
% (Portada/Apertura necesita logo pronto)
% -------------------------------------------------
\usepackage{graphicx}
\usepackage{xcolor}
\usepackage{tikz}
\usetikzlibrary{decorations.pathreplacing}

% Logo (digital/impresión)
\newif\ifdigital
\digitalfalse                          % por defecto: impresión B/N
% \digitaltrue                         % descomentar para versión digital (color)

\newcommand{\MHLogoColor}{media/Apertura/Logo.png}
\newcommand{\MHLogoPrint}{media/Apertura/Logo.pdf}

\newcommand{\MHLogo}[1][0.12\linewidth]{%
  \ifdigital
    \includegraphics[width=#1]{\MHLogoColor}%
  \else
    \IfFileExists{\MHLogoPrint}{\includegraphics[width=#1]{\MHLogoPrint}}{\includegraphics[width=#1]{\MHLogoColor}}%
  \fi
}

% Atajo: logo al pie, centrado (para apertura/portada)
\newcommand{\MHLogoBottom}[1][0.14\linewidth]{%
  \vfill
  \MHLogo[#1]%
  \vspace*{1.2cm}%
}

% -------------------------------------------------
% 3) ENCABEZADOS Y PIES (fancyhdr)
% (usa \href→ hyperref ya cargado)
% -------------------------------------------------
\usepackage{fancyhdr}
\pagestyle{fancy}
\fancyhf{}
\renewcommand{\headrulewidth}{0pt}

% Macro para título de capítulo en pie
\newcommand{\chaptertitle}{} % se actualiza en \fullchapter

% Páginas pares → número + web
\fancyfoot[LE]{%
  \rule[0.5ex]{\linewidth}{0.4pt}\\
  \small \thepage \quad \textit{visita} $\rightarrow$ \href{https://www.mathhoffmann.com}{www.mathhoffmann.com}%
}

% Páginas impares → símbolo+capítulo + número
\fancyfoot[RO]{%
  \rule[0.5ex]{\linewidth}{0.4pt}\\
  \small \chaptertitle \enspace \thepage%
}

% Estilos vacíos para portada/creditos/etc.
\usepackage{etoolbox} % utilidades varias
\fancypagestyle{emptyplain}{%
  \fancyhf{}%
  \renewcommand{\headrulewidth}{0pt}%
  \renewcommand{\footrulewidth}{0pt}%
}
\fancypagestyle{plain}{%
  \fancyhf{}%
  \renewcommand{\headrulewidth}{0pt}%
  \fancyfoot[RO]{\rule[0.5ex]{\linewidth}{0.4pt}\\ \small \chaptertitle \enspace \thepage}%
}

% -------------------------------------------------
% 4) TITULACIÓN (chapters/sections) — Aparición temprana
% -------------------------------------------------
\usepackage{titlesec}

% Bloque de capítulo (símbolo + título) y marcas para pies
% Uso: \fullchapter{Nº}{Símbolo}{Título}
\newcommand{\fullchapter}[3]{%
  \chapter*{}%
  \addcontentsline{toc}{chapter}{#3}%
  \markboth{#2\; #3}{}%
  \renewcommand{\chaptertitle}{#2\; #3}%
  \begin{center}
    \vspace*{-2cm} % ajuste fino vertical
    {\fontsize{22}{24}\selectfont\bfseries #1}\par\vspace{1ex}
    {\fontsize{20}{22}\selectfont\bfseries \scalebox{1.4}{#2} \quad #3}\par\vspace{0.2ex}
    \rule{\linewidth}{1pt}\\[-1.2ex]
    \rule{0.85\linewidth}{0.5pt}
  \end{center}%
}

% Subtítulo de capítulo (15pt bold)
\newcommand{\subtitulocapitulo}[1]{%
  \vspace{1.5ex}%
  \noindent{\fontsize{15}{17}\selectfont\bfseries #1}\par\vspace{1ex}%
}

% Secciones / Subsecciones
\titleformat{\section}
  {\normalfont\bfseries\fontsize{14}{16}\selectfont}{\thesection}{1.5em}{}
\titlespacing*{\section}{0pt}{1.5ex}{1ex}

\titleformat{\subsection}
  {\normalfont\bfseries\fontsize{12}{14}\selectfont}{\thesubsection}{1em}{}
\titlespacing*{\subsection}{0pt}{1ex}{0.5ex}

% -------------------------------------------------
% 5) INTRO DE CAPÍTULO — listas y cita manual
% (la cita ya la haces en intro.tex con flushright; aquí solo helpers)
% -------------------------------------------------
\usepackage{enumitem,pifont}
\setlist[itemize]{itemsep=0.3ex, topsep=0.5ex}

% Lista “¿Qué vas a aprender?”
\newlist{aprende}{itemize}{1}
\setlist[aprende]{label=\ding{228}, leftmargin=2em, itemsep=.25em, topsep=.6em}

% +1pt para la cita (adaptativo al tamaño actual)
\makeatletter
\newcommand{\oneptup}{%
  \edef\mh@size{\strip@pt\dimexpr\f@size pt + 1pt\relax}%
  \edef\mh@base{\strip@pt\dimexpr\f@size pt + 3pt\relax}%
  \fontsize{\mh@size}{\mh@base}\selectfont
}
\makeatother

% Macros mínimos de portada (si los usas en apertura)
\newcommand{\MHSizeSub}[1]{{\itshape\fontsize{15}{18}\selectfont #1}}
\newcommand{\MHSizeAuthor}[1]{{\fontsize{15}{18}\selectfont #1}}
\newcommand{\AuthorSC}[1]{{\textsc{#1}}}

% -------------------------------------------------
% 6) MATEMÁTICAS, ENTORNOS Y ESPACIADOS
% -------------------------------------------------
\usepackage{mathtools}
\usepackage{amsthm}
\usepackage{empheq}
\usepackage{siunitx}     % números/decimales
\usepackage{tasks}       % listas tipo ejercicios en columnas
\usepackage{array,tabularx}
\usepackage{multicol}
\usepackage{relsize}
\usepackage{soul}        % subrayados
\usepackage{changepage}  % márgenes locales

% Espaciado vertical en ecuaciones (compacto)
\makeatletter
\g@addto@macro\normalsize{%
  \setlength\abovedisplayskip{3pt}%
  \setlength\belowdisplayskip{3pt}%
  \setlength\abovedisplayshortskip{2pt}%
  \setlength\belowdisplayshortskip{2pt}%
}
\makeatother

% Notación rápida en TOC (símbolos grandes)
\newcommand{\tocmath}[1]{{\Large $#1$}}

% Números y decimales (ES/EN)
\sisetup{
  output-decimal-marker = {,}, % decimal = coma
  group-separator = {.},       % miles = punto
  group-minimum-digits = 4
}
\newcommand{\numes}[1]{%
  \begingroup
  \sisetup{output-decimal-marker = {,}, group-separator = {}, group-minimum-digits = 4}%
  \num{#1}%
  \endgroup
}
\newcommand{\numen}[1]{%
  \begingroup
  \sisetup{output-decimal-marker = {.}, group-separator = {,}}%
  \ensuremath{\num{#1}}%
  \endgroup
}

% Notación decimal periódica
\newcommand{\periodo}[1]{\overline{#1}}
\newcommand{\periodoen}[1]{\overline{#1}}
\newcommand{\periodico}[2]{%
  \numes{#1}\;#2\;#2\;#2\ldots \;=\; \numes{#1}\overline{#2}%
}

% -------------------------------------------------
% 7) CAJAS Y BLOQUES (tcolorbox) — usados en cuerpo
% -------------------------------------------------
\usepackage[most]{tcolorbox}
\tcbuselibrary{theorems,skins}

% Quote redefinido (estilo suave)
\renewenvironment{quote}{%
  \begin{tcolorbox}[
    enhanced,
    colback=black!5,
    colframe=white,
    boxrule=0pt,
    left=2cm, right=2cm,
    sharp corners,
    top=4pt, bottom=4pt,
    fontupper=\itshape
  ]%
}{%
  \end{tcolorbox}
}

% Bloques varios
\newenvironment{reseñaplana}{\par\noindent\ignorespaces}{\par}
\newenvironment{reseñaitem}{%
  \renewcommand\labelitemi{\ding{228}}%
  \begin{itemize}[leftmargin=2em]%
}{\end{itemize}}
\newenvironment{introblock}[1]{%
  \vspace{2ex}\noindent\textbf{\MakeUppercase{#1}}%
  \begin{quote}\itshape
}{\end{quote}\vspace{2ex}}
\newenvironment{temablock}{%
  \begin{tcolorbox}[colback=white,colframe=black!40,boxrule=0.4pt,
    sharp corners, enhanced, breakable, title={Explicación}]%
}{\end{tcolorbox}}
\newenvironment{cierreblock}{%
  \begin{tcolorbox}[colback=black!5,colframe=black!80,boxrule=0.8pt,
    sharp corners, enhanced, breakable, title={Reflexión final}]%
}{\end{tcolorbox}}

% Caja apertura de capítulo (si se usa)
\newtcolorbox{capitulobox}{%
  enhanced, breakable,
  colback=yellow!10, colframe=black, boxrule=0.6pt,
  arc=6pt, fontupper=\itshape, width=\textwidth,
  before skip=12pt, after skip=24pt,
  before upper={\begin{center}\begin{minipage}{0.9\textwidth}\centering\sloppy},
  after upper={\end{minipage}\end{center}}
}

% Caja de definición
\newtcolorbox{defbox}{%
  enhanced, breakable,
  colback=white, colframe=black!60, boxrule=0.6pt, arc=6pt,
  left=8pt, right=8pt, top=8pt, bottom=8pt,
  before skip=10pt, after skip=10pt,
  width=0.9\textwidth, center
}

% Caja extra/aparte
\newtcolorbox{extrabox}{%
  enhanced, breakable,
  colback=white, colframe=black!70, boxrule=0.8pt, arc=4pt,
  width=0.95\textwidth, center,
  left=10pt, right=10pt, top=10pt, bottom=10pt,
  before skip=16pt, after skip=16pt
}

% -------------------------------------------------
% 8) EJERCICIOS — bloque + entorno unificado
% -------------------------------------------------

% Contador global (igual que ya tenías)
\newcounter{ejercicioGlobal}
\renewcommand{\theejercicioGlobal}{\arabic{ejercicioGlobal}}

% Bloque decorativo para enmarcar un conjunto de ejercicios
\NewDocumentEnvironment{BloqueEjercicios}{ O{} }{%
  \refstepcounter{ejercicioGlobal}%
  \begin{tcolorbox}[
    enhanced,
    colback=white, colframe=black!60, boxrule=0.6pt,
    rounded corners, arc=2pt,
    left=1em, right=1em, top=1em, bottom=1em,
    title={Ejercicio~\theejercicioGlobal \IfNoValueTF{#1}{}{:\ #1}},
    fonttitle=\bfseries, coltitle=black,
    attach boxed title to top left={yshift=-2mm, xshift=2mm},
    boxed title style={colback=black!10, colframe=black!60, sharp corners,
      boxrule=0.4pt, top=0.6mm, bottom=0.6mm, left=1mm, right=1mm}
  ]%
}{%
  \end{tcolorbox}%
}

% Helpers compartidos (seguros y locales)
\providecommand{\task}{\item}             % por si alguien usa \task fuera
\providecommand{\contline}[1]{\\#1}       % línea de continuación
\providecommand{\MathItems}{}             % opcional: convierte \item en modo display si lo activas
\providecommand{\MHcolsep}{1.8em}         % separación estándar entre columnas

% ===== Entorno UNIFICADO de EJERCICIOS =====
% Uso: \begin{ejercicios}[<cols>][<opciones enumitem>]
%   <cols> por defecto = 1; si pones 2,3,4 activa multicol
%   <opciones enumitem> (opcional): por si quieres ajustar labelsep, noitemsep, etc.
\NewDocumentEnvironment{ejercicios}{ O{1} O{} }{%
  \setlength{\columnsep}{\MHcolsep}%
  \ifnum#1>1\begin{multicols}{#1}\fi
  \begin{enumerate}[label=\arabic*), leftmargin=1.8em, itemsep=1.2ex, topsep=1ex, parsep=1ex, #2]%
    \let\task\item
    \MathItems
}{%
  \end{enumerate}%
  \ifnum#1>1\end{multicols}\fi
}

% ---- Wrappers de compatibilidad (no toques capítulos) ----
% Definimos (no "renovamos") los atajos hacia el entorno principal 'ejercicios'
\NewDocumentEnvironment{ej1col}{}{ \begin{ejercicios}[1] }{ \end{ejercicios} }
\NewDocumentEnvironment{ej2col}{}{ \begin{ejercicios}[2] }{ \end{ejercicios} }
\NewDocumentEnvironment{ej3col}{}{ \begin{ejercicios}[3] }{ \end{ejercicios} }
\NewDocumentEnvironment{ej4col}{}{ \begin{ejercicios}[4] }{ \end{ejercicios} }

% Alias "automático": 2 columnas por defecto (para compatibilidad con tu código)
\NewDocumentEnvironment{ejAuto}{}{ \begin{ejercicios}[2] }{ \end{ejercicios} }
\NewDocumentEnvironment{ejcol}{}{  \begin{ejercicios}[2] }{ \end{ejercicios} }

% -------------------------------------------------
% 9) EJEMPLOS — entorno unificado (multicol + enumitem)
% -------------------------------------------------

% ===== Entorno UNIFICADO de EJEMPLOS =====
% Uso: \begin{ejemplos}[<cols>][<label>][<opciones enumitem>]
%   <cols>   : 1 por defecto (si >1 activa multicol)
%   <label>  : por defecto \textbullet  (pon (\alph*) si quieres a), b), c)…)
%   <opciones enumitem> : ajustes finos (opcional)
\NewDocumentEnvironment{ejemplos}{ O{1} O{\textbullet} O{} }{%
  \setlength{\columnsep}{\MHcolsep}%
  \ifnum#1>1\begin{multicols}{#1}\fi
  \begin{enumerate}[label=#2, labelsep=0.6em,
                    leftmargin=3.5em, itemsep=1.6ex, topsep=2ex, #3]%
    \let\task\item
}{%
  \end{enumerate}%
  \ifnum#1>1\end{multicols}\fi
}

% ---- Wrappers de compatibilidad (no toques capítulos) ----
% 1 columna
\NewDocumentEnvironment{ejemplos1}{}{ \begin{ejemplos}[1][(\alph*)] }{ \end{ejemplos} }
\NewDocumentEnvironment{ejemplos1bullet}{}{ \begin{ejemplos}[1][\textbullet] }{ \end{ejemplos} }

% 2 columnas
\NewDocumentEnvironment{ejemplos2short}{}{ \begin{ejemplos}[2][(\alph*)] }{ \end{ejemplos} }
\NewDocumentEnvironment{ejemplos2shortbull}{}{ \begin{ejemplos}[2][\textbullet] }{ \end{ejemplos} }
\NewDocumentEnvironment{ejemplos2large}{}{ \begin{ejemplos}[2][(\alph*)] }{ \end{ejemplos} }

% 3 columnas
\NewDocumentEnvironment{ejemplos3}{}{ \begin{ejemplos}[3][(\alph*)] }{ \end{ejemplos} }
\NewDocumentEnvironment{ejemplos3bullet}{}{ \begin{ejemplos}[3][\textbullet] }{ \end{ejemplos} }



% -------------------------------------------------
% 10) CASOS Y TABLAS AUXILIARES
% -------------------------------------------------
\newcounter{caso}[chapter]
\renewcommand{\thecaso}{\arabic{caso}}

\NewDocumentEnvironment{Caso}{ O{} }{%
  \refstepcounter{caso}%
  \par\medskip
  \noindent
  \tikz[baseline=(X.base)]{
    \node[
      fill=yellow!35, rounded corners=1pt,
      inner xsep=4pt, inner ysep=3pt,
      text height=2ex, text depth=.5ex, anchor=base west
    ] (X) {\textbf{\Large Caso \thecaso}%
      \if\relax\detokenize{#1}\relax\else~–~\textbf{#1}\fi};
  }%
  \par\vspace{1em}
}{%
  \par\vspace{1em}
}

\newcommand{\casolinea}[2]{%
  \noindent\hspace*{1.5em}%
  \begin{tabularx}{0.92\linewidth}{@{}X@{\hspace{1.5em}}>{\raggedright\arraybackslash}m{0.35\linewidth}@{}}%
    #1 & $\displaystyle #2$\\[0.8ex]%
  \end{tabularx}\par
}

% -------------------------------------------------
% 11) UTILIDADES GENERALES
% -------------------------------------------------
% Esquemas/figuras centradas con título opcional
\newcommand{\EsquemaConjuntos}[2][]{%
  \vspace{0.5cm}
  \begin{center}
    \includegraphics[width=0.9\linewidth]{#2}%
    \if\relax\detokenize{#1}\relax\else\\[4pt]\textbf{#1}\fi
  \end{center}
  \vspace{0.5cm}
}

% Fracciones en línea (ligeramente elevadas)
\newcommand{\nfrac}[3][2pt]{%
  \raisebox{#1}{$\dfrac{#2}{#3}$}\vphantom{p_{qgj}}%
}

% Llave decorativa inferior con texto
\newcommand{\llaveabajo}[2]{%
  \mathrel{\tikz[baseline=(m.base)]{
    \node[inner sep=0pt] (m) {$#1$};
    \draw [decorate,decoration={brace,mirror,amplitude=6pt}]
      (m.south west) -- (m.south east)
      node[midway,yshift=-9pt]{\scriptsize #2};
  }}%
}

% Estilo global de tablas
\renewcommand{\arraystretch}{1.2}
\setlength{\tabcolsep}{0.8em}

% Casos particulares (ej.: años, cap. 1)
\newcommand{\anios}{\text{años}}

% =============================
% FIN DE style.tex
% =============================


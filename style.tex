% !TEX root = main.tex
% =============================
% style.tex — Estilo del libro
% Proyecto: Math Hoffmann — Vol. 2
% Compilación: XeLaTeX (latexmk -xelatex)
% =============================

% -------------------------------------------------
% 0) MOTOR, IDIOMA, TIPOGRAFÍA Y MATEMÁTICAS (ROBUSTO)
% -------------------------------------------------

% ======= Build switches (modo borrador/producción, externalización TikZ) =======
\newif\ifdraftbuild      % \draftbuildtrue   -> borrador (más rápido)
\draftbuildfalse         % \draftbuildfalse  -> producción
\newif\ifexttikz         % externalizar TikZ (requiere -shell-escape)
\exttikzfalse

% ======= Motor tipográfico y matemático =======
\usepackage{fontspec}          % XeLaTeX/LuaLaTeX
\usepackage{mathtools}         % (carga amsmath)
\usepackage{unicode-math}      % Matemáticas Unicode

% ======= Idioma: SOLO ESPAÑOL =======
\usepackage{polyglossia}
\setdefaultlanguage{spanish}

% Helpers de contenido (solo español activo)
\newcommand{\ES}[1]{#1}
\newcommand{\EN}[1]{}

% Etiquetas (solo español)
\newcommand{\ChapterName}{Capítulo}

% ======= Fuentes con fallback seguro =======
\setmainfont{TeX Gyre Termes}  % Texto
\setmathfont{STIX Two Math}    % Matemáticas
\IfFontExistsTF{TeX Gyre Heros}{\setsansfont{TeX Gyre Heros}}{}
\IfFontExistsTF{TeX Gyre Cursor}{\setmonofont{TeX Gyre Cursor}}{}

% ======= Microtipografía y espaciado editorial =======
\usepackage{microtype}
\SetTracking{encoding=*}{40}       % tracking suave
\frenchspacing                     % espaciado tras punto consistente

% Párrafo: sin sangría + espacio entre párrafos
\setlength{\parindent}{0pt}
\setlength{\parskip}{0.8em}

% Control viudas/huérfanas (suaviza composiciones largas)
\clubpenalty=10000   % evita viudas
\widowpenalty=10000  % evita huérfanas
\displaywidowpenalty=10000


% -------------------------------------------------
% 1) PÁGINA, COLORES, GRÁFICOS, ENLACES
% -------------------------------------------------

% Geometría de página (ajusta márgenes a tu estándar)
\usepackage[letterpaper,margin=1in]{geometry}

% Colores amplios (tablas incluidas)
\usepackage[dvipsnames,svgnames,table]{xcolor}

% Gráficos y rutas
\usepackage{graphicx}
\graphicspath{{media/}{media/ch01/}{media/ch02/}} % añade más según crezcan

% Gráficos y rutas
\usepackage{graphicx}
\graphicspath{{media/}{media/ch01/}{media/ch02/}} % añade más según crezcan

% --- Dibujo y TikZ (necesario para llaves/decorations) ---
\usepackage{tikz}
\usetikzlibrary{calc,arrows.meta,positioning,decorations.pathreplacing}
% Si en algún punto usas marcas/flows, puedes añadir:
% \usetikzlibrary{decorations.markings}

% Modo borrador para imágenes (rápido en builds largos)
\ifdraftbuild
  \PassOptionsToPackage{draft}{graphicx}
\fi

% Hipervínculos y metadatos PDF (cargar al final del preámbulo)
\usepackage[hidelinks]{hyperref}
\hypersetup{
  pdftitle   = {Math Hoffmann — Volumen},
  pdfauthor  = {Jorge Gid Hoffmann / J&J Socrates, LLC},
  pdfsubject = {Matemáticas — Educación Secundaria},
  pdfkeywords= {Matemáticas, Bilingüe, Educación},
  pdfcreator = {XeLaTeX},
  unicode    = true
}

% Modo borrador para imágenes (rápido en builds largos)
\ifdraftbuild
  \PassOptionsToPackage{draft}{graphicx}
\fi

% Hipervínculos y metadatos PDF (cargar al final del preámbulo)
\usepackage[hidelinks]{hyperref}
\hypersetup{
  pdftitle   = {Math Hoffmann — Volumen},
  pdfauthor  = {Jorge Gid Hoffmann / J&J Socrates, LLC},
  pdfsubject = {Matemáticas — Educación Secundaria},
  pdfkeywords= {Matemáticas, Bilingüe, Educación},
  pdfcreator = {XeLaTeX},
  unicode    = true
}

% Si usas cleveref más adelante, cárgalo aquí (opcional):
% \usepackage[nameinlink]{cleveref}


% -------------------------------------------------
% 2) NÚMEROS, LISTAS Y UTILIDADES (sin 'locale')
% -------------------------------------------------

\usepackage{siunitx}

\sisetup{
  output-decimal-marker = {,}, % coma decimal
  group-separator       = {.}, % punto de millares
  per-mode              = symbol,
  detect-all
}

\usepackage{enumitem}
\setlist{topsep=.5ex,itemsep=.25ex,parsep=0pt,partopsep=0pt}

% (Opcional) Citas “inteligentes”
% \usepackage{csquotes} % si lo activas, usa \enquote{...}


% -------------------------------------------------
% 3) ENCABEZADOS Y PIES (fancyhdr)
% (usa \href→ hyperref ya cargado)
% -------------------------------------------------
\usepackage{fancyhdr}
\setlength{\headheight}{14pt} % evita warnings aunque no haya encabezado
\pagestyle{fancy}
\fancyhf{}
\renewcommand{\headrulewidth}{0pt}

% URL centralizada (mantener en un solo sitio)
\providecommand{\siteurl}{https://www.mathhoffmann.com}

% Macro para título de capítulo en pie (se actualiza en \fullchapter)
\newcommand{\chaptertitle}{} 

% Páginas pares → número + web
\fancyfoot[LE]{%
  \rule[0.5ex]{\linewidth}{0.4pt}\\
  \small \thepage \quad \textit{visita} → \href{\siteurl}{www.mathhoffmann.com}%
}

% Páginas impares → símbolo+capítulo + número
\fancyfoot[RO]{%
  \rule[0.5ex]{\linewidth}{0.4pt}\\
  \small \chaptertitle \enspace \thepage%
}

% Estilos vacíos para portada/créditos/etc.
\usepackage{etoolbox} % utilidades varias (ok tenerlo aquí)
\fancypagestyle{emptyplain}{%
  \fancyhf{}%
  \renewcommand{\headrulewidth}{0pt}%
  \renewcommand{\footrulewidth}{0pt}%
}
\fancypagestyle{plain}{%
  \fancyhf{}%
  \renewcommand{\headrulewidth}{0pt}%
  \fancyfoot[RO]{\rule[0.5ex]{\linewidth}{0.4pt}\\ \small \chaptertitle \enspace \thepage}%
}

% -------------------------------------------------
% 4) TITULACIÓN (chapters/sections) — Aparición temprana
% -------------------------------------------------
\usepackage{titlesec}

% Bloque de capítulo (símbolo + título) y marcas para pies
% Uso: \fullchapter{Nº}{Símbolo}{Título}
\newcommand{\fullchapter}[3]{%
  \clearpage% asegura corte de página y vacía floats pendientes
  \chapter*{}% mantiene el estilo 'plain' sin numeración visible
  \refstepcounter{chapter}% avanza el contador (resetea secciones/figs/tabs/eqs atados a chapter)
  \phantomsection% crea ancla para hyperref antes de la entrada en el TOC
  \addcontentsline{toc}{chapter}{#3}% entrada limpia en el índice
  \markboth{#2\; #3}{}% encabezados/pies (leftmark/rightmark)
  \renewcommand{\chaptertitle}{#2\; #3}% macro usada en el pie (fancyhdr)
  % (opcional) etiqueta de capítulo para referencias cruzadas
  \label{ch:\thechapter}%
  \begin{center}
    \vspace*{-2cm}% ajuste fino vertical (como lo tenías)
    {\fontsize{22}{24}\selectfont\bfseries #1}\par\vspace{1ex}
    {\fontsize{20}{22}\selectfont\bfseries \scalebox{1.4}{#2} \quad #3}\par\vspace{0.2ex}
    \rule{\linewidth}{1pt}\\[-1.2ex]
    \rule{0.85\linewidth}{0.5pt}
  \end{center}%
}

% Subtítulo de capítulo (15pt bold)
\newcommand{\subtitulocapitulo}[1]{%
  \vspace{1.5ex}%
  \noindent{\fontsize{15}{17}\selectfont\bfseries #1}\par\vspace{1ex}%
}

% Secciones / Subsecciones (mismo look & feel)
\titleformat{\section}
  {\normalfont\bfseries\fontsize{14}{16}\selectfont}{\thesection}{1.5em}{}
\titlespacing*{\section}{0pt}{1.5ex}{1ex}

\titleformat{\subsection}
  {\normalfont\bfseries\fontsize{12}{14}\selectfont}{\thesubsection}{1em}{}
\titlespacing*{\subsection}{0pt}{1ex}{0.5ex}


% -------------------------------------------------
% 5) INTRO DE CAPÍTULO — listas y cita manual (reforzado)
% -------------------------------------------------
\usepackage{enumitem,pifont}

% Lista global básica (no cambia tu look fuera de 'aprende')
\setlist[itemize]{itemsep=0.3ex, topsep=0.5ex, parsep=0pt, partopsep=0pt}

% Lista “¿Qué vas a aprender?”
\newlist{aprende}{itemize}{1}
\setlist[aprende]{%
  label=\ding{228},     % mismo símbolo que usas
  leftmargin=2em,
  itemsep=.25em,
  topsep=.6em,
  labelsep=.6em
}

% +1pt para la cita (adaptativo al tamaño actual)
\makeatletter
\newcommand{\oneptup}{%
  \edef\mh@size{\strip@pt\dimexpr\f@size pt + 1pt\relax}%
  \edef\mh@base{\strip@pt\dimexpr\f@size pt + 3pt\relax}%
  \fontsize{\mh@size}{\mh@base}\selectfont
}
\makeatother

% Macros mínimos de portada (si los usas en apertura)
\newcommand{\MHSizeSub}[1]{{\itshape\fontsize{15}{18}\selectfont #1}}
\newcommand{\MHSizeAuthor}[1]{{\fontsize{15}{18}\selectfont #1}}
\newcommand{\AuthorSC}[1]{{\textsc{#1}}}

% (Opcional, no invasivo) Entorno para la cita de apertura del capítulo
% Uso:
% \begin{introquote}[Autor, Obra]
%   Texto de la cita...
% \end{introquote}
\NewDocumentEnvironment{introquote}{O{} +b}{%
  \begin{flushright}\oneptup\itshape
}{%
  \par\smallskip
  {\normalfont\normalsize\upshape--- #1}%
  \end{flushright}%
}


% -------------------------------------------------
% 6) MATEMÁTICAS, ENTORNOS Y ESPACIADOS (robusto)
% -------------------------------------------------

% Paquetes (evitar dobles cargas y conflictos)
\usepackage{empheq}

% siunitx ya está cargado en el bloque 2; por seguridad, no lo recargamos aquí
% \usepackage{siunitx} % <- NO recargar para evitar avisos; usamos sus macros

\usepackage{tasks}       % listas de ejercicios en columnas
\usepackage{multicol}
\usepackage{relsize}
\usepackage{soul}        % subrayados/realces (NO usar dentro de fórmulas)
\usepackage{changepage}  % márgenes locales
\usepackage{array,tabularx}
% \usepackage{amsmath}   % <- NO cargar: ya viene vía mathtools (bloque 0)

% Espaciado vertical en ecuaciones (compacto y estable)
\makeatletter
\g@addto@macro\normalsize{%
  \setlength\abovedisplayskip{3pt}%
  \setlength\belowdisplayskip{3pt}%
  \setlength\abovedisplayshortskip{2pt}%
  \setlength\belowdisplayshortskip{2pt}%
}
\makeatother

% Notación rápida dentro del TOC (símbolos grandes)
\newcommand{\tocmath}[1]{{\Large $#1$}}

% ---------- siunitx helpers (coherentes con el bloque 2) ----------
% Formato general se fija en el BLOQUE 2:
%   \sisetup{output-decimal-marker = {,}, group-separator = {.}, ...}

% Español forzado (coma decimal) en contexto puntual
\newcommand{\numes}[1]{%
  \begingroup
  \sisetup{
    output-decimal-marker = {,}, % coma decimal
    group-separator = {.},       % miles con punto
    group-minimum-digits = 4
  }%
  \num{#1}%
  \endgroup
}

% Inglés forzado (punto decimal) en contexto puntual
\newcommand{\numen}[1]{%
  \begingroup
  \sisetup{
    output-decimal-marker = {.}, % punto decimal
    group-separator = {,}
  }%
  \num{#1}%
  \endgroup
}

% Notación decimal periódica
\newcommand{\periodo}[1]{\overline{#1}}     % 1.\periodo{23}
\newcommand{\periodoen}[1]{\overline{#1}}   % alias si luego activas EN
\newcommand{\periodico}[2]{% \periodico{1,}{23} -> 1,\overline{23}
  \numes{#1}\,\overline{#2}%
}

% ---------- Ajustes útiles para 'tasks' (no cambia tu look) ----------
% Etiqueta alineada y separación mínima entre ítems
\settasks{
  label-format = {\normalfont},
  label-align  = left,
  item-indent  = 1.25em,
  column-sep   = 1.2em,
  after-item-skip = .25ex
}

% -------------------------------------------------
% 7) CAJAS Y BLOQUES (tcolorbox) — usados en cuerpo
% -------------------------------------------------

% Cargar tcolorbox sólamente si no está cargado (evita dobles cargas)
\makeatletter
\@ifpackageloaded{tcolorbox}{}{%
  \usepackage[most]{tcolorbox} % skins, breakable, theorems, etc.
}
\makeatother
\tcbuselibrary{theorems,skins}

% -------------------------------------------------
% CITA (quote) — tcolorbox suave, con salto de página si es larga
% Uso: \begin{quote} Texto de la cita. \end{quote}
% Notas: margenes internos 2cm; cursiva; 'breakable' permite partir en páginas
\renewenvironment{quote}{%
  \begin{tcolorbox}[
    enhanced,
    breakable,
    colback=black!5,
    colframe=white,
    boxrule=0pt,
    left=2cm, right=2cm,
    sharp corners,
    top=4pt, bottom=4pt,
    fontupper=\itshape
  ]%
}{%
  \end{tcolorbox}
}

% -------------------------------------------------
% RESEÑA PLANA — bloque de texto simple sin marco
% Uso: \begin{reseñaplana} ... \end{reseñaplana}
\newenvironment{reseñaplana}{\par\noindent\ignorespaces}{\par}

% RESEÑA ÍTEMS — lista con viñeta \ding{228}
% Uso: \begin{reseñaitem} \item ... \end{reseñaitem}
\newenvironment{reseñaitem}{%
  \renewcommand\labelitemi{\ding{228}}%
  \begin{itemize}[leftmargin=2em]%
}{\end{itemize}}

% INTRO BLOCK — encabezado alto + cita estilizada (reutiliza 'quote')
% Uso: \begin{introblock}{Título} ... \end{introblock}
\newenvironment{introblock}[1]{%
  \vspace{2ex}\noindent\textbf{\MakeUppercase{#1}}%
  \begin{quote}\itshape
}{\end{quote}\vspace{2ex}}

% -------------------------------------------------
% TEMA BLOCK — caja de explicación (título opcional)
% Uso: \begin{temablock} ... \end{temablock}
%      \begin{temablock}[Otro título] ... \end{temablock}
\newenvironment{temablock}[1][Explicación]{%
  \begin{tcolorbox}[%
    colback=white, colframe=black!40, boxrule=0.4pt,
    sharp corners, enhanced, breakable,
    title={#1}%
  ]%
}{\end{tcolorbox}}

% CIERRE BLOCK — caja de cierre/reflexión (título opcional)
% Uso: \begin{cierreblock} ... \end{cierreblock}
%      \begin{cierreblock}[Conclusión] ... \end{cierreblock}
\newenvironment{cierreblock}[1][Reflexión final]{%
  \begin{tcolorbox}[%
    colback=black!5, colframe=black!80, boxrule=0.8pt,
    sharp corners, enhanced, breakable,
    title={#1}%
  ]%
}{\end{tcolorbox}}

% -------------------------------------------------
% CAPÍTULO BOX — caja ancha para apertura de capítulo
% Uso: \begin{capitulobox} Texto destacado de apertura... \end{capitulobox}
% Notas: centra el contenido, aumenta 2pt el tamaño de fuente dentro
\newtcolorbox{capitulobox}{%
  enhanced, breakable,
  colback=yellow!10, colframe=black, boxrule=0.6pt,
  arc=6pt, width=\textwidth,
  fontupper=\itshape,
  before skip=12pt, after skip=24pt,
  before upper={\begingroup\begin{center}\begin{minipage}{0.9\textwidth}\centering\sloppy\capituloboxPlusTwo},
  after upper={\end{minipage}\end{center}\endgroup}
}

% Helper de tamaño +2pt para 'capitulobox'
\makeatletter
\newcommand{\capituloboxPlusTwo}{%
  \fontsize{\dimexpr\f@size pt + 2pt\relax}{\baselineskip}\selectfont}
\makeatother

% -------------------------------------------------
% DEF BOX — definiciones (opción: pasar claves extra/título entre [])
% Uso: \begin{defbox} ... \end{defbox}
%      \begin{defbox}[title={Definición clave}] ... \end{defbox}
\newtcolorbox{defbox}[1][]{%
  enhanced, breakable,
  colback=white, colframe=black!60, boxrule=0.6pt, arc=6pt,
  left=8pt, right=8pt, top=8pt, bottom=8pt,
  before skip=10pt, after skip=10pt,
  width=0.9\textwidth, center,
  #1 % permite opciones opcionales sin romper usos previos
}

% EXTRA BOX — notas/apartes (opción: claves extra/título entre [])
% Uso: \begin{extrabox} ... \end{extrabox}
%      \begin{extrabox}[title={Ejemplo}] ... \end{extrabox}
\newtcolorbox{extrabox}[1][]{%
  enhanced, breakable,
  colback=white, colframe=black!70, boxrule=0.8pt, arc=4pt,
  width=0.95\textwidth, center,
  left=10pt, right=10pt, top=10pt, bottom=10pt,
  before skip=16pt, after skip=16pt,
  #1 % permite opciones opcionales sin romper usos previos
}


% -------------------------------------------------
% 8) EJERCICIOS — bloque + entorno unificado (robusto)
% -------------------------------------------------

% Contador del bloque de ejercicios (tal como lo tenías)
\newcounter{ejercicioGlobal}
\renewcommand{\theejercicioGlobal}{\arabic{ejercicioGlobal}}
% (Si algún día quieres reiniciarlo por capítulo: activar junto a 'chngcntr')
% \counterwithin{ejercicioGlobal}{chapter}
% \renewcommand{\theejercicioGlobal}{\thechapter.\arabic{ejercicioGlobal}}

% -------------------------------------------------
% BLOQUE DECORATIVO QUE ENVUELVE UN CONJUNTO DE EJERCICIOS
% Uso:
%   \begin{BloqueEjercicios}[Título opcional]
%     ... (uno o varios entornos 'ejercicios' dentro)
%   \end{BloqueEjercicios}
% Notas:
%   - Incrementa 'Ejercicio N' automáticamente en el título de la caja.
%   - Ideal para agrupar varias tandas (1 col, 2 col, etc.) bajo un encabezado.
\NewDocumentEnvironment{BloqueEjercicios}{ O{} }{%
  \refstepcounter{ejercicioGlobal}%
  \begin{tcolorbox}[
    enhanced,
    colback=white, colframe=black!60, boxrule=0.6pt,
    rounded corners, arc=2pt,
    left=1em, right=1em, top=1em, bottom=1em,
    title={Ejercicio~\theejercicioGlobal \IfNoValueTF{#1}{}{:\ #1}},
    fonttitle=\bfseries, coltitle=black,
    attach boxed title to top left={yshift=-2mm, xshift=2mm},
    boxed title style={colback=blue!15, colframe=black!60, sharp corners,
      boxrule=0.4pt, top=0.6mm, bottom=0.6mm, left=1mm, right=1mm}
  ]%
}{%
  \end{tcolorbox}%
}

% -------------------------------------------------
% HELPERS (locales al entorno; no tocamos el global)
% \contline{...} → salto de línea dentro de \item cuando conviene partir
\providecommand{\contline}[1]{\\#1}
% Separación estándar entre columnas de ejercicios
\providecommand{\MHcolsep}{1.8em}

% -------------------------------------------------
% ENTORNO UNIFICADO DE EJERCICIOS
% Uso básico:
%   \begin{ejercicios}               ... \end{ejercicios}      % 1 columna
%   \begin{ejercicios}[2]            ... \end{ejercicios}      % 2 columnas
%   \begin{ejercicios}[3][noitemsep] ... \end{ejercicios}      % 3 col + opciones enumitem
% Variante 'estrella' (opcional, no rompe usos): todos los ítems en display math
%   \begin{ejercicios}* [2]          ... \end{ejercicios}
%   (no uses '*' salvo que cada \item sea una expresión matemática en bloque)
%
% Detalles:
%   - Etiquetas 1), 2), 3) ... (cambia en #2 si quieres otro estilo).
%   - Pasa opciones de enumitem en el 2º opcional: p.ej. [start=11], [noitemsep], [label=\alph*)], etc.
\NewDocumentEnvironment{ejercicios}{ s O{1} O{} }{%
  \setlength{\columnsep}{\MHcolsep}%
  \ifnum#2>1\begin{multicols}{#2}\raggedcolumns\fi% multicol si pides >1
  \begin{enumerate}[label=\arabic*), leftmargin=1.8em, itemsep=1.2ex, topsep=1ex, parsep=1ex, #3]%
    % Dentro del entorno, \task actúa como \item (no definimos \task global)
    \let\task\item
    % Si usas la forma con '*', cada \item se envuelve en display math
    \IfBooleanT{#1}{\let\olditem\item
      \renewcommand{\item}{\olditem\begingroup\[\displaystyle}%
    }%
}{%
  % Cierre de display math si usaste la forma con '*'
  \IfBooleanT{#1}{\]\endgroup}%
  \end{enumerate}%
  \ifnum#2>1\end{multicols}\fi
}

% -------------------------------------------------
% WRAPPERS DE COMPATIBILIDAD (NO TOCAR LLAMADOS EN CAPÍTULOS)
% Uso:
%   \begin{ej1col} ... \end{ej1col}
%   \begin{ej2col} ... \end{ej2col}
%   \begin{ej3col} ... \end{ej3col}
%   \begin{ej4col} ... \end{ej4col}
%   \begin{ejAuto} ... \end{ejAuto}   % alias 2 columnas
%   \begin{ejcol}  ... \end{ejcol}    % alias 2 columnas
\NewDocumentEnvironment{ej1col}{}{ \begin{ejercicios}[1] }{ \end{ejercicios} }
\NewDocumentEnvironment{ej2col}{}{ \begin{ejercicios}[2] }{ \end{ejercicios} }
\NewDocumentEnvironment{ej3col}{}{ \begin{ejercicios}[3] }{ \end{ejercicios} }
\NewDocumentEnvironment{ej4col}{}{ \begin{ejercicios}[4] }{ \end{ejercicios} }
\NewDocumentEnvironment{ejAuto}{}{ \begin{ejercicios}[2] }{ \end{ejercicios} }
\NewDocumentEnvironment{ejcol}{}{  \begin{ejercicios}[2] }{ \end{ejercicios} }


% -------------------------------------------------
% 9) EJEMPLOS — entorno unificado (multicol + enumitem)
% -------------------------------------------------

% Entorno UNIFICADO de EJEMPLOS
% Uso base:
%   \begin{ejemplos}                ... \end{ejemplos}             % 1 col, bullets
%   \begin{ejemplos}[2][(\alph*)]   ... \end{ejemplos}             % 2 col, a) b) c)
%   \begin{ejemplos}[3][(\roman*)]  ... \end{ejemplos}             % 3 col, i) ii) iii)
% Opciones extra (3er parámetro): cualquier opción de enumitem, p.ej. [start=5], [noitemsep], etc.
%
% Variante estrella (opcional):
%   \begin{ejemplos}* [2][(\alph*)] ... \end{ejemplos}
%   → Cada \item se envuelve en display math automáticamente (útil si TODOS son fórmulas).
%
% Parámetros:
%   #1 (cols):   1 por defecto; si >1 activa multicol
%   #2 (label):  \textbullet por defecto (puedes pasar (\alph*), (\roman*), etc.)
%   #3 (opts):   opciones enumitem adicionales (opcionales)
\NewDocumentEnvironment{ejemplos}{ s O{1} O{\textbullet} O{} }{%
  \setlength{\columnsep}{\MHcolsep}%
  \ifnum#2>1\begin{multicols}{#2}\raggedcolumns\fi
  \begin{enumerate}[%
      label=#3,                 % etiqueta (bullet / (\alph*) / (\roman*), etc.)
      labelsep=0.6em,
      leftmargin=*,             % enumitem calcula margen adecuado según label
      itemsep=1.6ex,
      topsep=2ex,
      #4                        % opciones extra de enumitem
  ]%
    % Alias interno para comodidad (sin contaminar global): \task == \item
    \let\task\item
    % Si usas la forma con '*', cada \item será display math (no mezclar con texto corrido)
    \IfBooleanT{#1}{\let\olditem\item
      \renewcommand{\item}{\olditem\begingroup\[\displaystyle}%
    }%
}{%
  % Cierra el display del último \item si se usó la forma con '*'
  \IfBooleanT{#1}{\]\endgroup}%
  \end{enumerate}%
  \ifnum#2>1\end{multicols}\fi
}

% ---- Wrappers de compatibilidad (no tocar llamados en capítulos) ----
% 1 columna
%   \begin{ejemplos1} ... \end{ejemplos1}          → a) b) c) en 1 columna
%   \begin{ejemplos1bullet} ... \end{ejemplos1bullet} → bullets en 1 columna
\NewDocumentEnvironment{ejemplos1}{}{ \begin{ejemplos}[1][(\alph*)] }{ \end{ejemplos} }
\NewDocumentEnvironment{ejemplos1bullet}{}{ \begin{ejemplos}[1][\textbullet] }{ \end{ejemplos} }

% 2 columnas
%   \begin{ejemplos2short} ... \end{ejemplos2short}     → a) b) c) en 2 col
%   \begin{ejemplos2shortbull} ... \end{ejemplos2shortbull} → bullets en 2 col
%   \begin{ejemplos2large} ... \end{ejemplos2large}     → alias (mismo layout); si quieres, aquí puedes pasar [noitemsep] localmente
\NewDocumentEnvironment{ejemplos2short}{}{ \begin{ejemplos}[2][(\alph*)] }{ \end{ejemplos} }
\NewDocumentEnvironment{ejemplos2shortbull}{}{ \begin{ejemplos}[2][\textbullet] }{ \end{ejemplos} }
\NewDocumentEnvironment{ejemplos2large}{}{ \begin{ejemplos}[2][(\alph*)] }{ \end{ejemplos} }

% 3 columnas
%   \begin{ejemplos3} ... \end{ejemplos3}              → a) b) c) en 3 col
%   \begin{ejemplos3bullet} ... \end{ejemplos3bullet}  → bullets en 3 col
\NewDocumentEnvironment{ejemplos3}{}{ \begin{ejemplos}[3][(\alph*)] }{ \end{ejemplos} }
\NewDocumentEnvironment{ejemplos3bullet}{}{ \begin{ejemplos}[3][\textbullet] }{ \end{ejemplos} }

%   ELIMINANDO EL BULLET POINT

% En style.tex
\NewDocumentEnvironment{ejemplosplain}{ O{1} }%
  {\begin{ejemplos}[#1][\phantom{\textbullet}]}%
  {\end{ejemplos}}


% -------------------------------------------------
% 10) CASOS Y TABLAS AUXILIARES — versión final optimizada (robusta)
% Requiere: \usepackage{xparse,array}
% -------------------------------------------------

% Cargar TikZ solo si no está cargado (encabezado usa \tikz\node)
\makeatletter
\@ifpackageloaded{tikz}{}{\usepackage{tikz}}
\makeatother

% --- Contador de casos: reinicia en cada capítulo ---
\newcounter{caso}[chapter]
\renewcommand{\thecaso}{\arabic{caso}}

% ===== Parámetros de diseño =====
\newlength{\casoIndent}    \setlength{\casoIndent}{1.5em}  % margen común a la izquierda
\newlength{\casoLineW}     \setlength{\casoLineW}{\dimexpr\linewidth - \casoIndent\relax}

\newlength{\casoLeftW}     \setlength{\casoLeftW}{0.58\casoLineW} % ancho columna texto
\newlength{\casoColSep}    \setlength{\casoColSep}{1.25em}        % separación entre columnas
\newlength{\casoRightW}    \setlength{\casoRightW}{\dimexpr\casoLineW - \casoLeftW - \casoColSep\relax}

\newlength{\casoWsep}      \setlength{\casoWsep}{0.6\baselineskip} % aire en \casolineaW

% ===== Entorno Caso (ancho completo) =====
% Encabezado estilo "Caso N – Título" en franja amarilla
\NewDocumentEnvironment{CasoL}{ O{} }{%
  \refstepcounter{caso}%
  \par\medskip
  \noindent
  \tikz[baseline=(X.base)]{
    \node[
      fill=yellow!35, rounded corners=1pt,
      inner xsep=4pt, inner ysep=3pt,
      text height=2ex, text depth=.5ex, anchor=base west
    ] (X) {\textbf{\Large Caso \thecaso}%
      \if\relax\detokenize{#1}\relax\else~–~\textbf{#1}\fi};
  }%
  \par\vspace{1em}
}{%
  \par\vspace{1em}
}
% Alias
\NewDocumentEnvironment{Caso}{O{}}
  {\begin{CasoL}[#1]}
  {\end{CasoL}}

% ===== Línea ANCHO COMPLETO debajo del texto =====
% (desvincular y definir de cero para evitar conflictos)
\let\casolineaW\relax
% \casolineaW{texto}{expresión/figura}
% \casolineaW*{texto}{bloque ya en display (aligned, tikzpicture, etc.)}
\NewDocumentCommand{\casolineaW}{s m m}{%
  % Texto en UNA sola línea si cabe, ocupando todo el ancho útil del bloque:
  \noindent\hspace*{\casoIndent}\makebox[\casoLineW][l]{#2}\par
  % Aire configurable antes de la mates:
  \vspace{\casoWsep}%
  % Matemática a la izquierda del bloque (alineada con el margen):
  \noindent\hspace*{\casoIndent}%
  \IfBooleanTF{#1}{%
    #3%
  }{%
    \makebox[\casoLineW][l]{\(\displaystyle #3\)}%
  }%
  \par
}

% ===== Base en DOS COLUMNAS (alineación por la LÍNEA FINAL) =====
% b{...} de 'array' alinea por la última línea (renglón del ":")
\newcommand{\casolineaT}[2]{%
  \noindent\hspace*{\casoIndent}%
  \begin{tabular}[t]{@{}b{\casoLeftW}@{\hspace{\casoColSep}}b{\casoRightW}@{}}%
    #1 & #2\\
  \end{tabular}\par
}

% ===== Sabor de alto nivel (con/sin display automático) =====
\let\casolinea\relax
% \casolinea{texto}{expr}    -> expr inline con \displaystyle (IZQ), baseline OK
% \casolinea*{texto}{bloque} -> sin envolver (aligned/tikz/tabla). Para bloques multi-línea,
%                               puedes añadir [b] al entorno interno para apoyar en su última línea.
\NewDocumentCommand{\casolinea}{s m m}{%
  \IfBooleanTF{#1}{%
    \casolineaT{#2}{#3}%
  }{%
    \casolineaT{#2}{\(\displaystyle #3\)}%
  }%
}

% ===== Nota a TODO EL ANCHO (centrado opcional, sin cortes si cabe) =====
\let\casonota\relax
% \casonota{...}   -> izquierda
% \casonota*{...}  -> centrada
\NewDocumentCommand{\casonota}{s m}{%
  \noindent\hspace*{\casoIndent}%
  \IfBooleanTF{#1}{%
    \makebox[\casoLineW][c]{#2}%
  }{%
    \makebox[\casoLineW][l]{#2}%
  }%
  \par\vspace{0.6ex}%
}

% Atajo semántico para "Recuerda:" (centrado por defecto)
\providecommand{\casoRecuerda}{}% crea stub si no existía
\renewcommand{\casoRecuerda}[1]{%
  \casonota*{\textbf{Recuerda:} #1}%
}

\newcommand{\casolineaWc}[2]{%
  \noindent\hspace*{\casoIndent}\makebox[\casoLineW][l]{#1}\par
  \vspace{\casoWsep}%
  \noindent\hspace*{\casoIndent}\makebox[\casoLineW][c]{\(\displaystyle #2\)}\par
}

% Texto ancho completo con el mismo ritmo que \casolineaWc
\newcommand{\casolineaWtext}[1]{%
  \noindent\hspace*{\casoIndent}\makebox[\casoLineW][l]{#1}\par
  \vspace{\casoWsep}%
}

\newcommand{\CasosSetLeft}[1]{%
  \setlength{\casoLeftW}{#1\casoLineW}%
  \setlength{\casoRightW}{\dimexpr\casoLineW - \casoLeftW - \casoColSep\relax}%
}



% -------------------------------------------------
% 11) MACROS MATEMATICAS
% -------------------------------------------------

% --- Regla en caja + tag a 2 cm (estable) ---
\newlength{\ReglaTagGap}
\setlength{\ReglaTagGap}{2cm} % distancia entre la caja y el (n)

\providecommand{\ReglaCajaTag}{} % por si existía
\renewcommand{\ReglaCajaTag}[2]{% #1 = contenido SIN $, #2 = número
  \begin{center}
    {\setlength{\fboxsep}{6pt}\Large \Resultado{#1}}% caja un poco más grande y con más aire
    \hspace{\ReglaTagGap}%
    {\normalsize\bfseries (#2)}% tag +1 paso y en negrita
  \end{center}
}


% -------------------------------------------------
% 12) UTILIDADES GENERALES
% -------------------------------------------------

% Esquemas/figuras centradas con título opcional
% Uso: \EsquemaConjuntos[Subtítulo opcional]{ruta/archivo}
% Nota: no crea 'figure' ni \caption (no entra al LOF). Ideal para esquemas sueltos.
%       Si el archivo no existe, no rompe: deja un marcador silencioso.
\newcommand{\EsquemaConjuntos}[2][]{%
  \medskip
  \begingroup
    \centering
    \IfFileExists{#2}{%
      \includegraphics[width=.9\linewidth]{#2}%
    }{%
      \typeout{EsquemaConjuntos: archivo no encontrado: #2}%
      \setlength{\fboxsep}{1pt}%
      \fbox{\rule{.9\linewidth}{0pt}\rule{0pt}{.45\linewidth}}%
    }%
    \if\relax\detokenize{#1}\relax\else\\[4pt]\textbf{#1}\fi
  \par\endgroup
  \medskip
}

% Fracciones en línea ligeramente elevadas (para texto corrido)
% Uso: \nfrac{a}{b}  (eleva 2pt por defecto)   \nfrac[1pt]{a}{b} para ajuste fino
\newcommand{\nfrac}[3][2pt]{%
  \raisebox{#1}{$\dfrac{#2}{#3}$}\vphantom{p_{qgj}}%
}

% Llave decorativa inferior con texto (requiere TikZ + decorations.pathreplacing)
% Uso: \llaveabajo{EXPRESIÓN}{texto}
% Nota: si ves error de TikZ, asegúrate de tener: \usetikzlibrary{decorations.pathreplacing}
\newcommand{\llaveabajo}[2]{%
  \mathrel{\tikz[baseline=(m.base)]{%
    \node[inner sep=0pt] (m) {$#1$};
    \draw[decorate,decoration={brace,mirror,amplitude=6pt}]
      (m.south west) -- (m.south east)
      node[midway,yshift=-9pt]{\scriptsize #2};
  }}%
}

% Estilo global de tablas (un poco más de aire y col sep moderado)
% Efecto: aumenta alto de renglón y espacio horizontal entre columnas
\renewcommand{\arraystretch}{1.2}
\setlength{\tabcolsep}{0.8em}

% Unidad frecuente en textos (años) para usar dentro de fórmulas
% Uso: $20\,\anios$  (equivale a '20 años' manteniendo estilo)
\newcommand{\anios}{\text{años}}

% Línea de índice de capítulo con icono a la derecha (para páginas de apertura)
% Uso: \IndiceCap{Texto}{\iconPotenciacion}
\newcommand{\IndiceCap}[2]{%
  \noindent\textbf{#1}\hfill{\large\ensuremath{#2}}\\%
}

% Set de iconos (compactos) por capítulo — pensados para índice y encabezados
% Sugerencia: evita expresiones largas; mejor símbolos o formas cortas.
\newcommand{\iconIrracionales}{\pi}               % Cap. 1
\newcommand{\iconPotenciacion}{x^n}               % Cap. 2
\newcommand{\iconRadicacion}{\sqrt{\phantom{x}}}  % Cap. 3
\newcommand{\iconRacionalizacion}{\dfrac{1}{\sqrt{\phantom{x}}}} % Cap. 4
\newcommand{\iconCuadratica}{x^2}                 % Cap. 5
\newcommand{\iconSistemas}{\{\}}                  % Cap. 6
\newcommand{\iconInecuaciones}{x>a}               % Cap. 7
\newcommand{\iconValorAbsoluto}{|x|}              % Cap. 8
\newcommand{\iconPlano}{(x,y)}                    % Cap. 9
\newcommand{\iconFunciones}{f(x)}                 % Cap.10
\newcommand{\iconIneqCuad}{x^2>0}                 % Cap.11
\newcommand{\iconDiofanticas}{ax+by=c}            % Cap.12
\newcommand{\iconGeometria}{\triangle}            % Cap.13
\newcommand{\iconRecap}{\sum}                     % Cap.14

% --- Logo de portada: macro esperada por apertura/portada.tex ---
% Uso: \MHLogo[0.18\linewidth]   (ancho opcional; por defecto 0.18\linewidth)
% Ruta definida arriba; si falta el archivo, no rompe compilación (deja marcador)
\providecommand{\MHLogoPath}{media/Apertura/Logo.png}
\newcommand{\MHLogo}[1][0.18\linewidth]{%
  \IfFileExists{\MHLogoPath}{%
    \includegraphics[width=#1]{\MHLogoPath}%
  }{%
    \typeout{MHLogo: archivo no encontrado en \MHLogoPath}%
    \begingroup\setlength{\fboxsep}{1pt}%
      \fbox{\rule{#1}{0pt}\rule{0pt}{#1}}%
    \endgroup
  }%
}


% =============================
% FIN DE style.tex — UTILIDADES Y POLÍTICAS
% =============================

% -------------------------------------------------
% 11.A  CONSTANTES / DELIMITADORES / RESULTADOS
% -------------------------------------------------

% Constante de Euler (uso en texto y fórmulas): \ee
% Ej.: \(\ee^{i\pi}+1=0\)
\newcommand{\ee}{\mathrm{e}}

% Delimitadores "paired" con tamaño automático (requiere mathtools)
% Uso básico: \abs{x} → |x| ;   uso con tamaño auto: \abs*{\frac{a}{b}}
% Paréntesis con autoajuste: \paren*{\frac{a}{b}}
\DeclarePairedDelimiter\abs{\lvert}{\rvert}
\DeclarePairedDelimiter\paren{\lparen}{\rparen}

% Resultado enmarcado (para destacar un valor final en línea)
% Uso: \Resultado{-4}  →  [-4] enmarcado
\newcommand{\Resultado}[1]{\fbox{\(#1\)}}

% -------------------------------------------------
% 11.B  CAJAS ÚTILES DEL PROFESOR / NOTAS
% -------------------------------------------------

% Nota del profesor (tcolorbox): bloque transversal en tono azul suave
% Uso: \begin{ProfeTip}[title={Atajo}] ... \end{ProfeTip}
\newtcolorbox{ProfeTip}[1][]{enhanced, breakable,
  colback=blue!3, colframe=blue!55!black, boxrule=.5pt, arc=2pt,
  left=1em, right=1em, top=.6em, bottom=.6em,
  borderline west={2pt}{0pt}{blue!60},
  fonttitle=\bfseries, title={Nota del profesor},
  #1
}

% -------------------------------------------------
% 11.C  LISTAS ESPECIALES (IMPORTANCIA ACTUAL)
% -------------------------------------------------
% Listado exclusivo para “Importancia actual” (estilo propio)
% Requiere pifont (ya cargado en bloque 5).
% Uso: \begin{importancialist} \item ... \end{importancialist}
\newlist{importancialist}{itemize}{1}
\setlist[importancialist]{%
  label=\ding{51},     % ✔ (check)
  labelsep=.55em,      % separación etiqueta–texto
  leftmargin=*,        % alineación al margen
  itemsep=.35em,       % aire entre ítems
  topsep=.25em,        % aire con el párrafo previo
  parsep=0pt, partopsep=0pt,
  align=left
}

% -------------------------------------------------
% 11.D  ESTILO GLOBAL DE LISTAS GENÉRICAS
% -------------------------------------------------
% Nota: armonizado con el BLOQUE 5 para no sorprender cambios.
% Si en el futuro quieres otro “look” global, ajusta SOLO aquí.
\setlist[itemize]{%
  label=\textbullet,   % viñeta •
  leftmargin=*,        % alineado limpio al margen
  itemsep=0.3ex,       % mismo aire que bloque 5
  topsep=0.5ex,
  parsep=0pt, partopsep=0pt
}

% -------------------------------------------------
% 11.E  POLÍTICA EDITORIAL DE PUNTUACIÓN (Math Hoffmann)
% -------------------------------------------------
% No se coloca punto (.) ni signos de puntuación al final de
% expresiones matemáticas, ni en inline ($...$) ni en display (\[...\]).
% Mantener consistencia en todo el libro y la colección.
% (Se dejó una sola versión — la duplicada fue eliminada.)
% -------------------------------------------------

% -------------------------------------------------
% 11.F  PEGAMENTO / MICROAJUSTES
% -------------------------------------------------
% \IntroGlue: elimina el salto de párrafo que añade \parskip 
% (útil para “pegar” bloques con títulos o subtítulos sin comer espaciado global).
\newcommand{\IntroGlue}{\vspace{-\parskip}}






